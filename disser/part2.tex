\chapter{Моделирование морфологии и высокоэнергичного излучения W50}\label{SS433chapter}

\FloatBarrier
\section{Введение}

Туманность W50 - протяженная структура, видимая в радио лучах и создаваемая компактным источником SS 433 - двойной системой с аккрецирующей черной дырой (или нейтронной звездой) \cite{Margon1984, Fabrika2004, Cherepashchuk2025} показана на рисунке \ref{W50image}. Аккреция в данной системе вероятно в течение долгого времени поддерживается в сверх-эдингтоновском режиме \cite{Heuvel2017, Cherepashchuk2023}. Полная энергия, выделенная источником в окружающую среду сопоставима с энергией взрыва сверхновой. Поэтому SS 433 сильно воздействует на межзвездную среду, создавая туманность протяженностью больше 100 парсек \cite{Begelman1980}.

В процессе аккреции SS~433 производит транс-релятивистские джеты со скоростью потока $\sim 0.26\,c$ и углом растовора менее двух градусов \citep[e.g. ][]{Begelman1980, Calvani1981}. Эти узкие джеты видны в рентгеновском и оптическом диапазоне как пары линий сдвинутых в синюю и красную стороны, излучающиеся в пространственных областях $\lesssim 10^{12}$ см и $\lesssim 10^{15}$ см от источника соответственнои \citep[][]{Marshall2002, Fabrika2004,Khabibulin2016}. Однако сами джеты невидимы на расстояниях больших $\sim0.1$ пк от центрального источника \citep[e.g. ][]{Hjellming1981,Blundell2004}. 
Но так как на больших масштабах (от 40 до 110 парсек) окружающая SS~433 яркая радио и ${\rm H}\alpha$ туманность W50 имеет вытянутую структуру с соотношением 3:1 в направлении оси прецессии джета, предполагается, что именно джеты надувают туманность W50  \citep[][]{Begelman1980,Zealey1980,Eichler1983,Peter1993,Zavala2008, Panferov2017}. 

\begin{figure}[h]
	\centering
	\includegraphics[width=0.9\textwidth]{./img_part2/w50.png} 
	\caption{Изображение туманности W50 из работы \cite{Safi-Harb2022}. Красный цвет - радио \cite{Dubner1998}, зеленый - оптическое излучение \cite{Boumis2007}, желтый - мягкий рентген (0.5-1 кэВ), фиолетовый - средний рентген (1-2 кэВ), голубой - жесткий рентген (2-12 кэВ)\cite{Safi-Harb2022, Moldowan2005}.} 
	\label{W50image} 
\end{figure} 

Нет наблюдаемых свидетельств остановки джетов и в то же время наблюдаются широкие протяженные рентгеновские структуры по обе стороны от SS~433, сонаправленные с осью прецессии джетов \cite[e.g.,][]{Watson1983,Yamauchi1994,Safi-Harb1997,Safi-Harb1999,Brinkmann2007,Safi-Harb2022}. Эти расширенные джеты появляются на расстояниях около $\sim 20\,$пк от центрального источника и имеют угол раствора$\sim 20^\circ$, что в два раза меньше чем угол прецессии узких джетов. Рентгеновское излучение в этих структурах имеет, вероятно, синхротронное происхождение \citep[e.g.,][]{Brinkmann2007}, что подтверждается измерянной рентгеновской поляризацией телескопом IXPE \citep{IXPE2024SS433}. Высокоэнергичной гамма-излучения от источника предсказывалось многими моделями \citep[see, e.g.,][]{Safi-Harb1997, Reynoso2008, Aharonian1998, Kimura2020, Sudoh2020}. Недавно оно было задетектировано телескопами HAWC, H.E.S.S. и LHAASO \citep[][]{Abeysekara2018,HESS2024SS433,LHAASO2024SS433}, что указывает на наличие частиц, ускоренных до энергий порядка петаэлектрон-вольт. 

\begin{figure}[h]
	\centering
	\includegraphics[width=0.9\textwidth]{./img_part2/HESS.png} 
	\caption{Изображение туманности W50 в гамма лучах по данным телескопа HESS \cite{HESS2024SS433}. Приведена значимость сигнала в диапазонах: панель A - 0.8-2.5 ТэВ, панель B - 2.5-10 ТэВ, панель C - выше 10 ТэВ} 
	\label{W50HESS} 

	\centering
	\includegraphics[width=0.9\textwidth]{./img_part2/LHAASO.png} 
	\caption{Изображение туманности W50 в гамма лучах по данным телескопа LHAASO \cite{LHAASO2024SS433}. Приведена значимость сигнала в диапазонах: панель a - 1-25 ТэВ, панель b - 25-100 ТэВ, панель c - выше 100 ТэВ. Голубым цветом отмечена область излучающая гамма лучи с энергиями выше 10 ТэВ по данным HESS.} 
	\label{W50LHASSOimage} 
\end{figure} 

Модель, предложенная в работе \citep{Churazov}, объяснят структуру туманности W50 как результат взаимодействия мощного двухкомпонентного ветра, создаваемого аккреционным диском. Транс-релятивистский поток, сонаправленный с осью прецесси узких джетов взаимодействует с более медленным квазисферическим ветром. В результате этого взаимодействия в полярном ветре возникают реколлимационные ударные волны, приводящие к ускорению космических лучей и фысокоэнергичному излучению. В данной главе мы проведем моделирование структуры туманности W50 в рамках вышеописанной модели с помощью магнито-гидродинамического кода PLUTO \cite{Mignone2007}, моделирование ускорения космических лучей полученными ударными волнами, а так же проведем расчет высокоэнергичного излучений и сравнение с наблюдательными данными.

\section{Моделирование структуры туманности}
\section{Расчет высокоэнергичного излучения}
\subsection{Рентгеновское излучение}
\subsection{Гамма излучение}
\section{Выводы} \label{SS433conclision}


Получены следующие результаты:
\begin{enumerate}
	\item 
	\item 
\end{enumerate}


\clearpage