\chapter{Моделирование морфологии и высокоэнергичного излучения W50}\label{SS433chapter}

\FloatBarrier
\section{Введение}

Туманность W50 - протяженная структура, видимая в радио лучах и создаваемая компактным источником SS 433 - двойной системой с аккрецирующей черной дырой (или нейтронной звездой) \cite{Margon1984, Fabrika2004, Cherepashchuk2025} показана на рисунке \ref{W50image}. Аккреция в данной системе вероятно в течение долгого времени поддерживается в сверх-эдингтоновском режиме \cite{Heuvel2017, Cherepashchuk2023}. Полная энергия, выделенная источником в окружающую среду сопоставима с энергией взрыва сверхновой. Поэтому SS 433 сильно воздействует на межзвездную среду, создавая туманность протяженностью больше 100 парсек \cite{Begelman1980}.

В процессе аккреции SS~433 производит транс-релятивистские джеты со скоростью потока $\sim 0.26\,c$ и углом растовора менее двух градусов \citep[e.g. ][]{Begelman1980, Calvani1981}. Эти узкие джеты видны в рентгеновском и оптическом диапазоне как пары линий сдвинутых в синюю и красную стороны, излучающиеся в пространственных областях $\lesssim 10^{12}$ см и $\lesssim 10^{15}$ см от источника соответственнои \citep[][]{Marshall2002, Fabrika2004,Khabibulin2016}. Однако сами джеты невидимы на расстояниях больших $\sim0.1$ пк от центрального источника \citep[e.g. ][]{Hjellming1981,Blundell2004}. 
Но так как на больших масштабах (от 40 до 110 парсек) окружающая SS~433 яркая радио и ${\rm H}\alpha$ туманность W50 имеет вытянутую структуру с соотношением 3:1 в направлении оси прецессии джета, предполагается, что именно джеты надувают туманность W50  \citep[][]{Begelman1980,Zealey1980,Eichler1983,Peter1993,Zavala2008, Panferov2017}. 

\begin{figure}[h]
	\centering
	\includegraphics[width=0.9\textwidth]{./img_part2/w50.png} 
	\caption{Изображение туманности W50 из работы \cite{Safi-Harb2022}. Красный цвет - радио \cite{Dubner1998}, зеленый - оптическое излучение \cite{Boumis2007}, желтый - мягкий рентген (0.5-1 кэВ), фиолетовый - средний рентген (1-2 кэВ), голубой - жесткий рентген (2-12 кэВ)\cite{Safi-Harb2022, Moldowan2005}.} 
	\label{W50image} 
\end{figure} 

Нет наблюдаемых свидетельств остановки джетов и в то же время наблюдаются широкие протяженные рентгеновские структуры по обе стороны от SS~433, сонаправленные с осью прецессии джетов \cite[e.g.,][]{Watson1983,Yamauchi1994,Safi-Harb1997,Safi-Harb1999,Brinkmann2007,Safi-Harb2022}. Эти расширенные джеты появляются на расстояниях около $\sim 20\,$пк от центрального источника и имеют угол раствора$\sim 20^\circ$, что в два раза меньше чем угол прецессии узких джетов. Рентгеновское излучение в этих структурах имеет, вероятно, синхротронное происхождение \citep[e.g.,][]{Brinkmann2007}, что подтверждается измерянной рентгеновской поляризацией телескопом IXPE \citep{IXPE2024SS433}. Высокоэнергичной гамма-излучения от источника предсказывалось многими моделями \citep[see, e.g.,][]{Safi-Harb1997, Reynoso2008, Aharonian1998, Kimura2020, Sudoh2020}. Недавно оно было задетектировано телескопами HAWC, H.E.S.S. и LHAASO \citep[][]{Abeysekara2018,HESS2024SS433,LHAASO2024SS433}, что указывает на наличие частиц, ускоренных до энергий порядка петаэлектрон-вольт. 

\begin{figure}[h]
	\centering
	\includegraphics[width=0.9\textwidth]{./img_part2/HESS.png} 
	\caption{Изображение туманности W50 в гамма лучах по данным телескопа HESS \cite{HESS2024SS433}. Приведена значимость сигнала в диапазонах: панель A - 0.8-2.5 ТэВ, панель B - 2.5-10 ТэВ, панель C - выше 10 ТэВ} 
	\label{W50HESS} 

	\centering
	\includegraphics[width=0.9\textwidth]{./img_part2/LHAASO.png} 
	\caption{Изображение туманности W50 в гамма лучах по данным телескопа LHAASO \cite{LHAASO2024SS433}. Приведена значимость сигнала в диапазонах: панель a - 1-25 ТэВ, панель b - 25-100 ТэВ, панель c - выше 100 ТэВ. Голубым цветом отмечена область излучающая гамма лучи с энергиями выше 10 ТэВ по данным HESS.} 
	\label{W50LHASSOimage} 
\end{figure} 

Модель, предложенная в работе \citep{Churazov}, объяснят структуру туманности W50 как результат взаимодействия мощного двухкомпонентного ветра, создаваемого аккреционным диском. Транс-релятивистский поток, сонаправленный с осью прецесси узких джетов взаимодействует с более медленным квазисферическим ветром. В результате этого взаимодействия в полярном ветре возникают реколлимационные ударные волны, приводящие к ускорению космических лучей и фысокоэнергичному излучению. В данной главе мы проведем моделирование структуры туманности W50 в рамках вышеописанной модели с помощью магнито-гидродинамического кода PLUTO \cite{Mignone2007}, моделирование ускорения космических лучей полученными ударными волнами, а так же проведем расчет высокоэнергичного излучений и сравнение с наблюдательными данными.

\section{Моделирование структуры туманности} \label{W50morphology}
\begin{figure}
	\includegraphics[width=0.9\textwidth]{./img_part2/Scheme.pdf}
	\caption{Схематичное изображение системы SS433+W50 system adopted here. Яркие рентгеновские области в расширенном джете обозначены как `Cone', `Head', `e1' и `e2'. Они были обнаружени и обсуждались в работах \cite{Safi-Harb1997,Brinkmann2007,Safi-Harb2022}.}
	\label{Scheme}
\end{figure}


Туманность W50 имеет специфическую форму - у нее есть сферическая центральная часть и две вытянутые структуры к востоку и западу от ентрального источника. Рентгеновское излучение туманности неоднородно, а происходит от нескольких ярких областей - `e1", `e2", и `e3' на восточной стороне, `w1' и `w2' на западной \cite{Safi-Harb1997}. Следуя работе \cite{Churazov}, мы инициализируем течение из микроквазара в виде двух компонент - квазисферический медленный ветер и направленный транс-релятивистский полярный ветер (расширенный джет). Кинетическая мощность каждой компоненты составляет $2\times10^{39}$ erg/s, скорость медленного ветра - $\approx3000$ km/s, скорость полярного ветра $\approx0.2~c$. В результате взаимодействия направленного ветра, изотроного ветра и межзвездной среды образуются реколлимационные ударные волны в направленном ветре, что позволяет объяснить наличие нескольких ярких рентгеновских областей. Схема гидродинамической модели с излучающими областями, использованными в работах  \cite{Safi-Harb2022, IXPE2024SS433} показана на рисунке \ref{Scheme}.

Мы провели численное моделирование туманности W50 используя магнито-гидродинамический код PLUTO \cite{Mignone2007}. Мы использовали двумерные расчеты в цилиндрических координатах. В центральной области - сфере радиуса 5 пк, задавались внутренние граничные условия с заданными плотностью, давлением и скоростью радиального течения, которые зависели от полярного угла $\theta$ данной пространственной ячейки. Для малых углов параметры соответствовали транс-релятивистскому направленному потоку, а для больших - изотропному ветру. Внешняя межзвездная среда задавалась с постоянными плотностью и давлением.

Магнитное поле задавалось равным 100 $\mu$G в основании расширенного джета и 1 $\mu$G в окружающей среде и изотропном ветре, и направлено по оси z. В результате моделирования удалось получить протяженную структуру со сферической центральной частью и двумя сильными реколлимационными ударными волнами с заметными дисками Маха, расположенными на расстоянии примерно 32 и 48 парсек от центрального источника, что близко к наблюдаемым ярким рентгеновским областям `e1' и `e2' в восточном джете \cite{Safi-Harb1997}. Структура гидродинамического течения изображена на рисунке ~\ref{velocity} - крупномасштабная карта скорости на верхней панели и увеличенные карты скорости, плотности и магнитного поля в области между центральным источником и `e2' на нижних панелях. Для данного сетапа были выбраны следующие параметры: кинетическая можность изотропного ветра $P_i$ и равная ей кинетическая мощность расширенного джета $P_j = 2.1\times10^{39}$ эрг/с, скорость изотропного ветра  $v_i=3000$ км/с, скорость расширенного джета $v_j=0.2~c$, его угол полураствора $5^{\circ}$, плотность межзвездной среды $\rho_{amb} = 0.005~\rm{cm^{-3}}$ и ее температура $T_{amb} = 8\times10^4~\rm{K}$. Время моделирования составило 30 000 лет.

Подобную структуру можно и получить и при других параметрах среды, при большей плотности и меньшей температуре, например
 ($\rho_{amb}=0.05~\rm{cm^{-3}}$, $T_{amb} = 8.6\times10^3~\rm{K}$). Но в этом случае подобная структура с сильными реколлимационными ударными волнами, расположенными в местах соответствующих областям `e1' и `e2' будет наблюдаться на более поздних временах, примерно 100 000 лет. 

Первая реколлимационная ударная волна случается на расстоянии соответствующем ударной волне остановки изотропного ветра. Дальнейшее поведение расширенного джета - будет ли он распространяться с серией реколимационных ударных волн, или остановится после первой, определяется давлением внешней средыю Оно должно быть достаточно малым, чтобы направленный поток распространился на большие расстояния. К тому же, как показано в работе \cite{Marti2018}, относительно низкое внешнее давление необходимо для формирование перпендикулярных ударных волн (дисков Маха), а не системы косых ударных волн, менее пригодных для ускорения частиц. В дополнение, при численном моделировании необходимо достаточно высокое пространственное разрешение, чтобы было возможно наблюдать диски Маха.

Одномерные профили плотности, скорости и энтропии гидродинамического потока вблизи областей `e1' and `e2' показаны на рисунке~\ref{MHD_profile}. Скачок плотности на обоих ударных волнах близок к 4, что говорит о сильной ударной волне. Скачки энтропии тоже достаточно большие. Так же имеются признаки более слабых ударных волн с меньшими скачками плотности. Кинетическое Монте-Карло моделирование показывает, что при учете давления ускоренных частиц, сжатие на фронте ударной волны может значительно превосходить адиабатическиое предельное значение 4 (см. например \cite{Bykov3inst2014}).

\begin{figure}
	\includegraphics[width=0.9\textwidth]{./img_part2/velocity4.png}
	\caption{Моделированная структура туманности W50: a) крупномасштабная карта скорости b) карта скорости вблизи областей e1 и e2 c) карта плотности d) карта магнитного поля.} 
	\label{velocity}
\end{figure}


\begin{figure}
	\includegraphics[width=0.9\textwidth]{./img_part2/profile_1d_window2.png}
	\caption{Одномерные профили плотности, скорости и энтропии, соответствующие расчетам, показанным на рисунке~\ref{velocity}.} 
	\label{MHD_profile}
\end{figure}



\textcolor{red}{In the minimalists' scenario, the main goal was reproducing the overall morphology of the W50 radio emission and the emergence of the `extended X-ray jets' as a result of recollimation shocks in the anisotropic wind scenario. This model has several free parameters, including overall energetics, age of the system, density of the ambient medium, velocities and densities of the isotropic and polar winds, and the opening angle of the polar component. Various combinations of these parameters can lead to a qualitatively similar morphology of the nebula (see Fig.A1 in \cite{Churazov}). Here, our primary goal is to find parameters that lead to the efficient acceleration of the particles that give rise to the observed emission of EXJs. Knowing that accelerating electrons to above 100 TeV energies is especially efficient for strong shocks with velocities $\gtrsim$ 0.2c \cite{Churazov}, we ran the MHD version of a set-up with this value of the polar component velocity, but did not require the morphology of the simulated nebula boundary to match the observed one perfectly. We leave the task of finding a parameter combination that satisfies all observational requirements best for a future study.} 
\section{Расчет высокоэнергичного излучения}
Туманность W50 является ярким источником высокоэнергичного электромагнитного излучения. Она видна в рентгеновском диапазоне, с четко различимыми излучающими областями \cite{Safi-Harb1997, Safi-Harb2022}, а так же в гамма-излучении, кторое носит более диффузный характер \cite{HESS2024SS433, LHAASO2024SS433}. Для рождения фотонов с такими высокими энергиями (несколько сотен ТэВ) необходимо ускорение частиц до высоких энергий, порядка 1 ПэВ. Мы использовали спектры ускоренных частиц, полученные методом Монте-Карло моделирования, описанного в работах \citep[][]{EBJ96, Bykov3inst2014}, задавая в расчетах параметры ударных волн, соответствующие ударным волнам, полученным из гидродинамического моделирования. Использованный метод Монте-Карло моделирования не может корректно расчитывать ускорение электронов, но в диффузионной модели с коэффициентом диффузии зависящим только от заряда и энергии частицы движение электронов и протонов с одинаковыми энергиями будет одинаковым. Вследствие чего для достаточно больших энергий функции распределения протонов и электронов будут подобны. Функции распределения ускоренных частиц на фронтах ударных волн, с параметрами соответствующими ударным волнам, приведенным на рисунке \ref{velocity} показаны на рисунке \ref{pdf_shock_Q_esc}.

\begin{figure}
	\includegraphics[width=0.9\textwidth]{./img_part2/pdf_shock_Q_esc.pdf}
	\caption{Функции распределения ускоренных частиц на фронтах ударных волн, полученные Монте-Карло моделированием. Красная и зеленая кривая - протоны, синяя и фиолетовая - электроны. Пунктирные лини показывают низкоэнергичную часть электронной функции распределения, которая не может быть точно получена из метода Монте-Карло.} 
	\label{pdf_shock_Q_esc}
\end{figure}

\subsection{Рентгеновское излучение}
Рентгеновское излучение SS433/W50 имеет несколько особенностей - неоднородное пространственное распределение с несколькими яркими областями, спектр с разными степенными индексами в разных диапазонах и поляризацию. В этом разделе представлена модель объясняющая рентгеновское излучение от восточной части туманности W50 как синхротронное излучение электронов, ускоренных на транс-релятивистских ударных волнах, в неоднородном турбулентном магнитном поле. Предположение о неоднородности магнитного поля подтверждается тем, что узкие пространственные пики яркости излучения в области, обозначенной как `Head' в диапазонах энергии 0.3-10 кэВ и 3-30 кэВ \cite{Safi-Harb2022} имеют приблизительно одинаковую ширину, в то время как если бы ширина пиков определялась бы синхротронными потерями в однородном поле, они имели бы разную ширину.

\begin{figure}[th!]
	\includegraphics[width=0.9\textwidth]{./img_part2/W50synch.png}
	\caption{Моделированный спектр рентгеновского излучения в областях `Head' и `Cone' и наблюдательные данные из работы\cite{Safi-Harb2022}.} 
	\label{synchrotron}
\end{figure}

Полноценное описание переноса частиц в сложном гидродинамическом потоке, описанном в разделе \ref{W50morphology},  требует моделирования стохастического магнитного поля. Что в свою очередь требует расчета турбулентности создаваемой реколимационными ударными волнами, сдвиговыми течениями и током космическич лучей. Это задача недоступная численному моделированию на современных компьютерах, поэтому для расчета распространения ускоренных электронов внутри расширенного джета мы решаем одномерное конвекционно-диффузионное уравнение с добавленным членом для учета синхротронных потерь в заданном профиле усредненного магнитного поля \ref{convection-diffusion}. 

\begin{eqnarray}\label{convection-diffusion}
	&& \frac{\partial f({\bf r},E)}{\partial t} + {\bf v_{\alpha}(r)}\frac{\partial f({\bf r},E)}{\partial r_\alpha} = \nonumber \\ &=& \frac{\partial}{\partial r_\alpha} D_{\alpha \beta}({\bf r},E)\frac{\partial f({\bf r}, E)}{\partial r_\beta}  - \frac{\partial [a({\bf r},E) f({\bf r},E)]}{\partial E}.
\end{eqnarray}

Где $D_{\alpha \beta}({\bf r},E)$ - коэффициент диффузии,  $a({\bf r},E)=\frac{4\sigma_T}{3m_e^2c^3}\left(B({\bf r})^2/8\pi + U_{ph}\right)E^2$ - потери электронов на синхротронное и комптоновское излучение, $\sigma_T$ - томпсоновское сечение, $B$ - магнитное поле и $U_{ph}$ - плотность энергии фотонного поля. 

Чтобы еще упростить уравнение, мы оценили вклад диффузионного члена, проведя Монте-Карло моделирование распространения частиц в соответствии с уравнением \ref{convection-diffusion}. Было установлено, что для коэфициентов диффузии $D < 3\times 10^{28} см^2/c$, что в 100 раз меньше бомовского коэффициента для энергии 100 ТэВ, конвекционный перенос вдоль оси джета доминирует доминирует над диффузионным на интересующих нас масштабах. Таким образом, дифузионным членом можно пренебречь. Тогда в приближении стационарности уравнение \ref{convection-diffusion} можно решить аналитически.
Энергия частицы при движении вдоль оси z меняется как

\begin{equation}
	E(z) = \frac{E\left(0\right)}{1+E\left(0\right)k(z)}
\end{equation}

где $k(z) = \int_0^z \frac{4\sigma_T}{3m_e^2c^3}\left(B(z)^2/8\pi + U_{ph}\right) \frac{dx}{v(z)}$. 

А функция распределения при движения за фронт ударной волны определяется уравнением

\begin{equation}
	f(z,E)=f\left(0,\frac{E}{1-k(z)E}\right)\frac{1}{\left(1-k(z)E\right)^2}
\end{equation}

где $f(0,E)$ - функция распределения электронов на фронте ударной волны, показанная на рисунке~\ref{pdf_shock_Q_esc}. 

В качестве фотонного поля бралось реликтовое излучение, так как для интересующих нас наиболее высоких энрегий (выше 100 ТэВ), рассеяние на более высокочастотных фотонах подавлено в Клейн-Нишиновском режиме.

Магнитное поле, необходимое для сильного синхротронного излучения, обеспечивается за счет создания и усиления турбулентности ударной волной. Известно, что неустойивости, создаваемые током космических лучей, создают турбулентное поле в предфронте ударной волны \cite{Bell2004,Bykov3inst2014}. При переходе этого поля через фронт, оно дополнительно усиливается за счет сжатия на ударной волне \cite{Zirakashvili2008pol, Inoue13}. Для расчета синхротронного излучения мы брали пространственный профиль турбулентного магнитного поля за фронтом ударной волны в соответствии с процедурой, описанной в работе \cite{BykovTycho}. С помощью кода PLUTO исследовалось развитие мелкомасштабной неустойчивостей заданным током космических лучей, а потом прохождение созданной турбулентности через фронт ударной волны. Так же исследовалась крупномасштабная турбулентность, связанная с гидродиномическим течением. Полученные профили среднеквадратичного магнитного поля для различных видов турбулентности показаны на рисунках \ref{B2_Bell_art} и \ref{B2_all_longscale_art}.

\begin{figure}[h]
	\begin{minipage}[t]{0.49\textwidth}
		\centering
		\includegraphics[width=0.95\textwidth]{./img_part2/B2_Bell_art.pdf} 
		\caption{Профиль среднеквадратичных компонент магнитного поля при прохождении турбулентности, создаваемой Белловской неустойчивостью, через ударную волну.} 
		\label{B2_Bell_art}
	\end{minipage} \hfill
	\begin{minipage}[t]{0.49\textwidth}
		\centering
		\includegraphics[width=0.95\textwidth]{./img_part2/B2_all_longscale_art.pdf} 
		\caption{Профиль среднеквадратичных компонент магнитного поля при прохождении длинноволновой турбулентности через ударную волну.} 
		\label{B2_all_longscale_art}
	\end{minipage}
\end{figure}

Для объяснения наблюдаемых профилей рентгеновской яркости в близи первой реколлимационной ударной волны (областях `Head' и `Cone') мы использовали следующую многокомпонентную модель магнитного поля - сумма описанных выше двух турбулентных полей, с равными на фронте значениями магнитного поля $7.5$ мкГс и характерными масштабами $3\times10^{17}$ см для короткомасштабной неустойчивости и $3\times10^{19}$ см для крупномасштабной. Так же считалось, что среднеквадратичное перестает спадать с удалением от фронта при достижении значения $8$ мкГс. Скорость течения за фронтом ударной волны бралась равной $v = 10.3\times10^8$ см/с (согласованно с Монте-Карло моделированием спектра ускоренных частиц). Результаты моделирования синхротронного излучения, а так же наблюдаетельные данные \cite{Safi-Harb2022} представлены на рисунке \ref{synchrotron}. Полученные спектры согласуются с данными наблюдений

\begin{figure}[h!]
	\includegraphics[width=0.8\textwidth]{./img_part2/profile.png}
	\caption{Моделированные пространственные профили синхротронного излучения вблизи области e1 в диапазоне 0.3-10 кэВ и данные наблюдений XMM (верхняя панель), в диапазоне 3-30 кэВ и данные наблюдений NuSTAR (средняя панель) и профиль используемого магнитного поля (нижняя панель).} 
	\label{profile_e1}
\end{figure}

\begin{figure}[h!]
	\includegraphics[width=0.8\textwidth]{./img_part2/profile_Brinkmann.png}
	\caption{Моделированный пространственный профиль синхротронного излучения в диапазоне 0ю3-10 кэВ и данные наблюдений XMM (верхняя панель) и использованный профиль магнитного поля (нижняя панель).} 
	\label{profile_Brinkmann}
\end{figure}

Аналогично было проведено моделирование рентгеновского синхротронного излучения в области `e2', наблюдаемый пространственный профиль которого описан в работе \cite{Brinkmann2007}. В этом случае мы взяли турбулентное магнитное поле с максимальным значением $9$ мкГс на фронте ударной волны, постоянным уровнем $3$ мкГс, характерной длиной турбулентности $2\times10^{18}$ см и скоростью за фронтом $9.3\times10^8$ см/с. Результаты моделирования показаны на рисунке \ref{profile_Brinkmann}.

\subsection{Гамма излучение}
In this Section, we model the gamma radiation of W50, which is observed by H.E.S.S. \cite{HESS2024SS433} and LHAASO \cite{LHAASO2024SS433}. The leptonic model assumes that it is produced via inverse Compton scattering of electrons accelerated by recollimation shocks in the collimated outflow.  

In Fig. \ref{compton}, the red curve shows the gamma radiation produced within the outflow by accelerated VHE electrons as they are advected through the downstream of the first shock to the second one. This corresponds to pure advection of VHE electrons along the X-ray outflow with the turbulent magnetic field and the distribution function of electrons, which were used to simulate the synchrotron X-ray radiation discussed above. The red curve is significantly below the observational data, and it has a different spectral shape at low energies. This is because the radiating electrons in the X-ray jet are in the thin target regime.
On the other hand, the electrons accelerated at the first shock diffuse in the direction transverse to the X-ray outflow. These electrons left the axial X-ray outflow and diffused through the surrounding cocoon, comprising a spatially broad component. If their diffusion coefficient in the cocoon is $\sim 10^{28} см^2/c$ at 100 TeV, they would radiate gamma-rays in the thick target regime in a wide cocoon of a size $\sim$ 30 pc around the axial jet. Electrons that have flown through the X-ray axial jet with a magnetic field amplified by cosmic ray-driven instabilities at the shock and escape into the cocoon downstream would also radiate in the thick regime. In Fig. \ref{compton}, the gamma-ray spectrum integrated over the wide cocoon is shown by the blue curve, which is in reasonably good agreement with the LHAASO data points taken from \cite{LHAASO2024SS433}. The gamma-ray spectrum integrated over a somewhat narrower region of the cocoon corresponds to the intermediate regime between the thin and the thick targets. It is shown by a brown curve in Fig.~\ref{compton} and is compared with the data reported by H.E.S.S. \cite{HESS2024SS433}.     


The LHAASO images above 100 TeV \cite{LHAASO2024SS433} show a more extended emission region compared to the lower energy bands. This morphology can be understood within our diffusive shock acceleration model, where the highest energy particles escape into the shock upstream, where the low magnetic field allows the escaped VHE leptons to form an extended gamma-ray emission region by the inverse Compton process.  


In our model, $\sim$ 30 $\%$ of the electrons accelerated at the shock front had escaped from the jet and diffused in the cocoon. This allows us to explain the observational data from LHAASO as shown in Figure \ref{compton}. H.E.S.S. data fall between the model for LHAASO and radiation from the jet itself, which can be explained by a smaller region size, observed by H.E.S.S.

There are recent studies of the acceleration of PeV protons in microquasars \citep{2024A&A...691A..93A,2024arXiv241108762P,2025arXiv250510620K}.  A spherically symmetric semi-analytical model of particle acceleration to energies of several PeV by the termination wind shock was discussed in \cite{2024arXiv241108762P}. The acceleration of particles in the shear flows of jet-cocoon-structured microquasars is suggested in \citep{2025arXiv250620193Z}. Shear acceleration in the model requires the injection of particles pre-accelerated up to TeV regime energies. Hadronic interactions of accelerated protons are producing a substantial fraction of the observed VHE gamma-ray emission in the models \citep{2024arXiv241108762P,2025arXiv250510620K}. 

The recollimation shocks in the diffusive shock acceleration model considered here efficiently accelerate protons up to PeV energies, as is seen in Fig. \ref{pdf_shock_Q_esc}.
In our model, $\gsim$ 10\% of the shocked jet power is transferred to accelerated protons while the VHE emission comes from the inverse Compton radiation of VHE electrons accelerated at shocks, producing polarized synchrotron X-ray emission of the extended X-ray jet. The leptonic gamma-ray emission is produced in the thick target regime within the cocoon surrounding the extended X-ray jet. 
The VHE gamma rays can be produced by hadronic interactions of the protons with ambient matter. The very low number density of the plasma in the extended jet and in the surrounding cocoon would require a very small diffusion coefficient for the accelerated VHE protons to contribute to the detected gamma-ray emission from the cocoon, given by the gamma-ray significance maps of H.E.S.S. and LHAASO. Therefore, most of the accelerated protons would escape the accelerator in the SS433/W50 extended X-ray jet and then contribute to the gamma-ray emission from the denser gas in the vicinity of W50 and supply the galactic cosmic-ray population.   
\section{Выводы} \label{SS433conclision}


Получены следующие результаты:
\begin{enumerate}
	\item 
	\item 
\end{enumerate}


\clearpage