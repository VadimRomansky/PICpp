\chapter{Моделирование морфологии и высокоэнергичного излучения W50}\label{SS433chapter}

\FloatBarrier
\section{Введение}

Туманность W50 - протяженная структура, видимая в радио лучах и создаваемая компактным источником SS 433 - двойной системой с аккрецирующей черной дырой (или нейтронной звездой) \cite{Margon1984, Fabrika2004, Cherepashchuk2025} показана на рисунке \ref{W50image}. Аккреция в данной системе вероятно в течение долгого времени поддерживается в сверх-эдингтоновском режиме \cite{Heuvel2017, Cherepashchuk2023}. Полная энергия, выделенная источником в окружающую среду сопоставима с энергией взрыва сверхновой. Поэтому SS 433 сильно воздействует на межзвездную среду, создавая туманность протяженностью больше 100 парсек \cite{Begelman1980}.

В процессе аккреции SS~433 производит транс-релятивистские джеты со скоростью потока $\sim 0.26\,c$ и углом растовора менее двух градусов \citep[e.g. ][]{Begelman1980, Calvani1981}. Эти узкие джеты видны в рентгеновском и оптическом диапазоне как пары линий сдвинутых в синюю и красную стороны, излучающиеся в пространственных областях $\lesssim 10^{12}$ см и $\lesssim 10^{15}$ см от источника соответственнои \citep[][]{Marshall2002, Fabrika2004,Khabibulin2016}. Однако сами джеты невидимы на расстояниях больших $\sim0.1$ пк от центрального источника \citep[e.g. ][]{Hjellming1981,Blundell2004}. 
Но так как на больших масштабах (от 40 до 110 парсек) окружающая SS~433 яркая радио и ${\rm H}\alpha$ туманность W50 имеет вытянутую структуру с соотношением 3:1 в направлении оси прецессии джета, предполагается, что именно джеты надувают туманность W50  \citep[][]{Begelman1980,Zealey1980,Eichler1983,Peter1993,Zavala2008, Panferov2017}. 

\begin{figure}[h]
	\centering
	\includegraphics[width=0.9\textwidth]{./img_part2/w50.png} 
	\caption{Изображение туманности W50 из работы \cite{Safi-Harb2022}. Красный цвет - радио \cite{Dubner1998}, зеленый - оптическое излучение \cite{Boumis2007}, желтый - мягкий рентген (0.5-1 кэВ), фиолетовый - средний рентген (1-2 кэВ), голубой - жесткий рентген (2-12 кэВ)\cite{Safi-Harb2022, Moldowan2005}.} 
	\label{W50image} 
\end{figure} 

Нет наблюдаемых свидетельств остановки джетов и в то же время наблюдаются широкие протяженные рентгеновские структуры по обе стороны от SS~433, сонаправленные с осью прецессии джетов \cite[e.g.,][]{Watson1983,Yamauchi1994,Safi-Harb1997,Safi-Harb1999,Brinkmann2007,Safi-Harb2022}. Эти расширенные джеты появляются на расстояниях около $\sim 20\,$пк от центрального источника и имеют угол раствора$\sim 20^\circ$, что в два раза меньше чем угол прецессии узких джетов. Рентгеновское излучение в этих структурах имеет, вероятно, синхротронное происхождение \citep[e.g.,][]{Brinkmann2007}, что подтверждается измерянной рентгеновской поляризацией телескопом IXPE \citep{IXPE2024SS433}. Высокоэнергичной гамма-излучения от источника предсказывалось многими моделями \citep[see, e.g.,][]{Safi-Harb1997, Reynoso2008, Aharonian1998, Kimura2020, Sudoh2020}. Недавно оно было задетектировано телескопами HAWC, H.E.S.S. и LHAASO \citep[][]{Abeysekara2018,HESS2024SS433,LHAASO2024SS433}, что указывает на наличие частиц, ускоренных до энергий порядка петаэлектрон-вольт. 

\begin{figure}[h]
	\centering
	\includegraphics[width=0.9\textwidth]{./img_part2/HESS.png} 
	\caption{Изображение туманности W50 в гамма лучах по данным телескопа HESS \cite{HESS2024SS433}. Приведена значимость сигнала в диапазонах: панель A - 0.8-2.5 ТэВ, панель B - 2.5-10 ТэВ, панель C - выше 10 ТэВ} 
	\label{W50HESS} 

	\centering
	\includegraphics[width=0.9\textwidth]{./img_part2/LHAASO.png} 
	\caption{Изображение туманности W50 в гамма лучах по данным телескопа LHAASO \cite{LHAASO2024SS433}. Приведена значимость сигнала в диапазонах: панель a - 1-25 ТэВ, панель b - 25-100 ТэВ, панель c - выше 100 ТэВ. Голубым цветом отмечена область излучающая гамма лучи с энергиями выше 10 ТэВ по данным HESS.} 
	\label{W50LHASSOimage} 
\end{figure} 

Модель, предложенная в работе \citep{Churazov}, объяснят структуру туманности W50 как результат взаимодействия мощного двухкомпонентного ветра, создаваемого аккреционным диском. Транс-релятивистский поток, сонаправленный с осью прецесси узких джетов взаимодействует с более медленным квазисферическим ветром. В результате этого взаимодействия в полярном ветре возникают реколлимационные ударные волны, приводящие к ускорению космических лучей и фысокоэнергичному излучению. В данной главе мы проведем моделирование структуры туманности W50 в рамках вышеописанной модели с помощью магнито-гидродинамического кода PLUTO \cite{Mignone2007}, моделирование ускорения космических лучей полученными ударными волнами, а так же проведем расчет высокоэнергичного излучений и сравнение с наблюдательными данными.

\section{Моделирование структуры туманности}
\begin{figure}
	\includegraphics[width=0.9\textwidth]{./img_part2/Scheme.pdf}
	\caption{Схематичное изображение системы SS433+W50 system adopted here. Яркие рентгеновские области в расширенном джете обозначены как `Cone', `Head', `e1' и `e2'. Они были обнаружени и обсуждались в работах \cite{Safi-Harb1997,Brinkmann2007,Safi-Harb2022}.}
	\label{Scheme}
\end{figure}


Туманность W50 имеет специфическую форму - у нее есть сферическая центральная часть и две вытянутые структуры к востоку и западу от ентрального источника. Рентгеновское излучение туманности неоднородно, а происходит от нескольких ярких областей - `e1", `e2", и `e3' на восточной стороне, `w1' и `w2' на западной \cite{Safi-Harb1997}. Следуя работе \cite{Churazov}, мы инициализируем течение из микроквазара в виде двух компонент - квазисферический медленный ветер и направленный транс-релятивистский полярный ветер (расширенный джет). Кинетическая мощность каждой компоненты составляет $2\times10^{39}$ erg/s, скорость медленного ветра - $\approx3000$ km/s, скорость полярного ветра $\approx0.2~c$. В результате взаимодействия направленного ветра, изотроного ветра и межзвездной среды образуются реколлимационные ударные волны в направленном ветре, что позволяет объяснить наличие нескольких ярких рентгеновских областей. Схема гидродинамической модели с излучающими областями, использованными в работах  \cite{Safi-Harb2022, IXPE2024SS433} показана на рисунке \ref{Scheme}.

Мы провели численное моделирование туманности W50 используя магнито-гидродинамический код PLUTO \cite{Mignone2007}. Мы использовали двумерные расчеты в цилиндрических координатах. В центральной области - сфере радиуса 5 пк, задавались внутренние граничные условия с заданными плотностью, давлением и скоростью радиального течения, которые зависели от полярного угла $\theta$ данной пространственной ячейки. Для малых углов параметры соответствовали транс-релятивистскому направленному потоку, а для больших - изотропному ветру. Внешняя межзвездная среда задавалась с постоянными плотностью и давлением.

Магнитное поле задавалось равным 100 $\mu$G в основании расширенного джета и 1 $\mu$G в окружающей среде и изотропном ветре, и направлено по оси z. В результате моделирования удалось получить протяженную структуру со сферической центральной частью и двумя сильными реколлимационными ударными волнами с заметными дисками Маха, расположенными на расстоянии примерно 32 и 48 парсек от центрального источника, что близко к наблюдаемым ярким рентгеновским областям `e1' и `e2' в восточном джете \cite{Safi-Harb1997}. Структура гидродинамического течения изображена на рисунке ~\ref{velocity} - крупномасштабная карта скорости на верхней панели и увеличенные карты скорости, плотности и магнитного поля в области между центральным источником и `e2' на нижних панелях. Для данного сетапа были выбраны следующие параметры: кинетическая можность изотропного ветра $P_i$ и равная ей кинетическая мощность расширенного джета $P_j = 2.1\times10^{39}$ эрг/с, скорость изотропного ветра  $v_i=3000$ км/с, скорость расширенного джета $v_j=0.2~c$, его угол полураствора $5^{\circ}$, плотность межзвездной среды $\rho_{amb} = 0.005~\rm{cm^{-3}}$ и ее температура $T_{amb} = 8\times10^4~\rm{K}$. Время моделирования составило 30 000 лет.

Подобную структуру можно и получить и при других параметрах среды, при большей плотности и меньшей температуре, например
 ($\rho_{amb}=0.05~\rm{cm^{-3}}$, $T_{amb} = 8.6\times10^3~\rm{K}$). Но в этом случае подобная структура с сильными реколлимационными ударными волнами, расположенными в местах соответствующих областям `e1' и `e2' будет наблюдаться на более поздних временах, примерно 100 000 лет. 

Первая реколлимационная ударная волна случается на расстоянии соответствующем ударной волне остановки изотропного ветра. Дальнейшее поведение расширенного джета - будет ли он распространяться с серией реколимационных ударных волн, или остановится после первой, определяется давлением внешней средыю Оно должно быть достаточно малым, чтобы направленный поток распространился на большие расстояния. К тому же, как показано в работе \cite{Marti2018}, относительно низкое внешнее давление необходимо для формирование перпендикулярных ударных волн (дисков Маха), а не системы косых ударных волн, менее пригодных для ускорения частиц. В дополнение, при численном моделировании необходимо достаточно высокое пространственное разрешение, чтобы было возможно наблюдать диски Маха.

Одномерные профили плотности, скорости и энтропии гидродинамического потока вблизи областей `e1' and `e2' показаны на рисунке~\ref{MHD_profile}. Скачок плотности на обоих ударных волнах близок к 4, что говорит о сильной ударной волне. Скачки энтропии тоже достаточно большие. Так же имеются признаки более слабых ударных волн с меньшими скачками плотности. Кинетическое Монте-Карло моделирование показывает, что при учете давления ускоренных частиц, сжатие на фронте ударной волны может значительно превосходить адиабатическиое предельное значение 4 (см. например \cite{Bykov3inst2014}).

\begin{figure}
	\includegraphics[width=0.9\textwidth]{./img_part2/velocity4.png}
	\caption{Моделированная структура туманности W50: a) крупномасштабная карта скорости b) карта скорости вблизи областей e1 и e2 c) карта плотности d) карта магнитного поля.} 
	\label{velocity}
\end{figure}


\begin{figure}
	\includegraphics[width=0.9\textwidth]{./img_part2/profile_1d_window2.png}
	\caption{Одномерные профили плотности, скорости и энтропии, соответствующие расчетам, показанным на рисунке~\ref{velocity}.} 
	\label{MHD_profile}
\end{figure}



\textcolor{red}{In the minimalists' scenario, the main goal was reproducing the overall morphology of the W50 radio emission and the emergence of the `extended X-ray jets' as a result of recollimation shocks in the anisotropic wind scenario. This model has several free parameters, including overall energetics, age of the system, density of the ambient medium, velocities and densities of the isotropic and polar winds, and the opening angle of the polar component. Various combinations of these parameters can lead to a qualitatively similar morphology of the nebula (see Fig.A1 in \cite{Churazov}). Here, our primary goal is to find parameters that lead to the efficient acceleration of the particles that give rise to the observed emission of EXJs. Knowing that accelerating electrons to above 100 TeV energies is especially efficient for strong shocks with velocities $\gtrsim$ 0.2c \cite{Churazov}, we ran the MHD version of a set-up with this value of the polar component velocity, but did not require the morphology of the simulated nebula boundary to match the observed one perfectly. We leave the task of finding a parameter combination that satisfies all observational requirements best for a future study.} 
\section{Расчет высокоэнергичного излучения}
Туманность W50 является ярким источником высокоэнергичного электромагнитного излучения. Она видна в рентгеновском диапазоне, с четко различимыми излучающими областями \cite{Safi-Harb1997, Safi-Harb2022}, а так же в гамма-излучении, кторое носит более диффузный характер \cite{HESS2024SS433, LHAASO2024SS433}. Для рождения фотонов с такими высокими энергиями (несколько сотен ТэВ) необходимо ускорение частиц до высоких энергий, порядка 1 ПэВ. Мы использовали спектры ускоренных частиц, полученные методом Монте-Карло моделирования, описанного в работах \citep[][]{EBJ96, Bykov3inst2014}, задавая в расчетах параметры ударных волн, соответствующие ударным волнам, полученным из гидродинамического моделирования. Использованный метод Монте-Карло моделирования не может корректно расчитывать ускорение электронов, но в диффузионной модели с коэффициентом диффузии зависящим только от заряда и энергии частицы движение электронов и протонов с одинаковыми энергиями будет одинаковым. Вследствие чего для достаточно больших энергий функции распределения протонов и электронов будут подобны. Функции распределения ускоренных частиц на фронтах ударных волн, с параметрами соответствующими ударным волнам, приведенным на рисунке \ref{velocity} показаны на рисунке \ref{pdf_shock_Q_esc}.

\begin{figure}
	\includegraphics[width=0.9\textwidth]{./img_part2/pdf_shock_Q_esc.pdf}
	\caption{Функции распределения ускоренных частиц на фронтах ударных волн, полученные Монте-Карло моделированием. Красная и зеленая кривая - протоны, синяя и фиолетовая - электроны. Пунктирные лини показывают низкоэнергичную часть электронной функции распределения, которая не может быть точно получена из метода Монте-Карло.} 
	\label{pdf_shock_Q_esc}
\end{figure}

\subsection{Рентгеновское излучение}
X-ray emission from SS433/W50 nebula has several specific features - inhomogeneous spatial profile with bright regions, energy spectra, and polarization. In this section, we present a model describing observed X-ray emission from the eastern lobe of W50 \citep{Brinkmann2007,Safi-Harb2022}, via synchrotron radiation of electrons, accelerated on two trans-relativistic shocks, in the inhomogeneous anisotropic turbulent magnetic field. Assumption of inhomogeneous profile of turbulent magnetic field is supported by the fact that narrow peaks in observed profiles in the `Head' region in energy ranges 0.3-10 keV and 3-30 keV \cite{Safi-Harb2022} have similar widths, which can be explained by the structure of the magnetic field. On the contrary, under the assumption that the peak width is a result of synchrotron losses,  peaks should have different widths.

\begin{figure}[th!]
	\includegraphics[width=0.9\textwidth]{./img_part2/W50synch.png}
	\caption{Modeled spectrum of X-ray synchrotron radiation in `Head' and `Cone' regions and observational data from \cite{Safi-Harb2022}. The estimated model fluxes of the GHz synchrotron radio emission from the two regions are about a few mJy. } 
	\label{synchrotron}
\end{figure}



The exact description of VHE particle transport in the collimated outflow and the surrounding plasma cocoon would require a multiscale modeling of stochastic magnetic fields, which in turn would include different sources of turbulence associated with a complex MHD dynamics with recollimation shocks, shear flows, and cosmic-ray-driven instabilities. Such a model is unfeasible now. Instead, we parameterized the diffusion coefficient of relativistic particles in the system. 

To evaluate the electron distribution function in the downstream flow, we use the advection - diffusion equation, neglecting stochastic acceleration and adiabatic losses:


\begin{eqnarray}
	&& \frac{\partial f({\bf r},E)}{\partial t} + {\bf v_{\alpha}(r)}\frac{\partial f({\bf r},E)}{\partial r_\alpha} = \nonumber \\ &=& \frac{\partial}{\partial r_\alpha} D_{\alpha \beta}({\bf r},E)\frac{\partial f({\bf r}, E)}{\partial r_\beta}  - \frac{\partial [a({\bf r},E) f({\bf r},E)]}{\partial E}.
\end{eqnarray}







Here $D_{\alpha \beta}({\bf r},E)$ is the diffusion coefficient,  $a({\bf r},E)=\frac{4\sigma_T}{3m_e^2c^3}\left(B({\bf r})^2/8\pi + U_{ph}\right)E^2$ is the electron synchrotron and Compton loss rates, $\sigma_T$ is the Thompson cross section, $B$ is the magnetic field and $U_{ph}$ is the energy density of the photon field. 


We simulated a set of models with different assumptions about the diffusion coefficient of VHE particles.   If $D < 3\times 10^{28} см^2/c$, which is about 100 times larger than the Bohm diffusion coefficient at 100 TeV, the advection dominates the transport along the X-ray jet. Then, in a stationary regime, one can solve this equation by calculating the energy losses of the single electron during its advection into the downstream.

\begin{equation}
	E(z) = \frac{E\left(0\right)}{1+E\left(0\right)k(z)}
\end{equation}

where $k(z) = \int_0^z \frac{4\sigma_T}{3m_e^2c^3}\left(B(z)^2/8\pi + U_{ph}\right) \frac{dx}{v(z)}$. 
The photon field is taken as cosmic microwave background radiation, scattering on all other possible components is suppressed due to the Klein-Nishina regime in the considered energy range (electrons with energies higher than 100 TeV). The turbulent magnetic field was established according to Section \ref{sec:CorSh}. We used two models of the magnetic
field. In the first model, we calculate
only a small-scale turbulent magnetic field. 
In this model, we assume that the field decay stops at some distance
from the shock, so the magnetic field remains constant onward. 
In our calculation, we assume that the critical value at which the field decay stops is $8$ $\mu G$. In Model 2, we assume a superposition of the simulated short- and long-scale magnetic fields calculated
for $l_{*}=3\cdot10^{17}$~cm and $l_{*}=3\cdot10^{19}$~cm. 
Using the results of MHD magnetic field simulations discussed in Section \ref{sec:CorSh}, we constructed a smooth analytical approximation for square-averaged magnetic fields, neglecting small-scale fluctuations, as was done in \cite{2024PhRvD.110b3041B} for Tycho SNR. The averaging
is performed on slices of the simulated MHD datasets at fixed coordinates $z$. 
The square average magnetic field profiles for models 1 and 2
are shown in Fig.~\ref{fig:Bprof}. 
The approximation has a different
functional dependence on the distance from the shock for normal ($B_{\parallel}$) and transverse (${B_{\perp}}$) field components.

Thus, in the advection model, the electron distribution at a distance $z$ from the shock in the downstream can be evaluated as

\begin{equation}
	f(z,E)=f\left(0,\frac{E}{1-k(z)E}\right)\frac{1}{\left(1-k(z)E\right)^2}
\end{equation}

where $f(0,E)$ is the distribution function of electrons at the shock front, shown in Figure~\ref{pdf_shock_Q_esc}. 
The modeled level of synchrotron radiation derived with the electron-to-proton ratio fixed to be 3.3$ \times 10^{-3}$ at 1 GeV described above fits well with the observed data.   The energy spectrum of the `Head' and `Cone' regions in the short-scale Bell's turbulent field model with characteristic length $l_\ast = 3\times10^{17}~\rm{cm}$, maximum field on the front $15~\rm{\mu G}$ and constant level $8~\rm{\mu G}$, constant velocity $v = 10.3\times10^8~\rm{cm/s}$ (consistent with Monte Carlo simulation), and observational data \cite{Safi-Harb2022} are shown in Fig.~\ref{synchrotron}. The modeled spectra are consistent with the power-law functions obtained in the observations.

\begin{figure}[h!]
	\includegraphics[width=0.9\textwidth]{./img_part2/profile.png}
	\caption{Spatial profiles of synchrotron emission in e1 region in the energy range 0.3-10 keV predicted by the model in comparison with the XMM data \cite{Safi-Harb2022} (top panel), in the energy range 3-30 keV and the NuSTAR data \cite{Safi-Harb2022} (middle panel), and the profile of magnetic field used in the model (bottom panel).} 
	\label{profile_e1}
\end{figure}

\begin{figure}[h!]
	\includegraphics[width=0.9\textwidth]{./img_part2/profile_Brinkmann.png}
	\caption{Modeled spatial profile of synchrotron radiation in e2 region in energy range 0.3-10 keV and XMM data \cite{Brinkmann2007} (top panel), and profile of magnetic field used for modeling radiation (bottom panel). } 
	\label{profile_Brinkmann}
\end{figure}

Spatial profile of the X-ray luminosity for the same model is shown in Figure~\ref{profile_e1}. It is also consistent with observations. The profile for the `e2' region is described in \cite{Brinkmann2007}, and was studied only by XMM. To explain this case, we use a turbulent field with the maximum value of $9~\rm{\mu G}$, constant level of $3~\rm{\mu G}$, characteristic length $2\times10^{18}~\rm{cm}$ and velocity $9.3\times10^8~\rm{cm/s}$, results of simulation are shown in Fig.~\ref{profile_Brinkmann}.

\begin{figure}[h!]
	\includegraphics[scale=0.4]{B_test_model1}
	\includegraphics[scale=0.4]{B_test_model2}
	
	\caption{\label{fig:Bprof} The upper panel shows the downstream rms magnetic field spatial profile approximations for model 1 described in the text, while
		the lower panel shows it for model 2. }
	
\end{figure}
\subsection{Гамма излучение}
\section{Выводы} \label{SS433conclision}


Получены следующие результаты:
\begin{enumerate}
	\item 
	\item 
\end{enumerate}


\clearpage