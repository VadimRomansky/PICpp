\chapter{Моделирование излучения FBOT}\label{FBOT}

Последние наблюдения открыли новый класс объектов - быстрые голубые оптические транзиенты (FBOT)\cite{Drout2014, Margutti2014, Coppejans2020, Ho2020}. Вместе со слабыми гамма-всплесками они возможно принадлежат к переходному классу объектов между нерелятивистскими сверхновыми и обычными гамма-всплесками. Как показано в работе Шевалье и Ирвина \cite{Chevalier2011} наблюдаемые характеристики вспышки определяются длительностью действия центрального источника энергии выброса и его взаимодействием со внешними слоями взрывающейся звезды. При различных параметрах вспышка будет относиться к разным классам объектов. Известные на данный момент быстрые голубые оптические транзиенты AT2018cow \cite{Margutti2014}, CSS161010 \cite{Coppejans2020}, ZTF2018abvkwla \cite{Ho2020} and AT2020xnd \cite{Ho2021, Bright2021} характеризуются большой светимостью $L > 10^43 эрг/с$, коротким характерным временным масштабом порядка нескольких дней, низкой выброшенной массой эжекты и высокими скоростями.
Рассматриваемы в данной главе объект CSS161010, расположенный в карликовой галактике на расстоянии 150 мегапарсек имеет по оценкам следующие\cite{Coppejans2020} характеристики: скорость эжекты 0.55 c на ранних этапах и около 0.3 c на поздних (порядка одного года), выброшенная масса порядка 0.01-0.1 солнечных масс. 
%почему четвертый был отдельно в статье?


В данной главе будет проведено Particle-in-Cell моедлирование ускорения частиц в транс-релятивистской ударной волне со скоростями, характерными для FBOT объектов. По имеющимся распределениям будет расчитано синхротронное излучение и с помощью фитирования наблюдательных данных будут подобраны такие параметры как магнитное поле, концентрация вещества и геометрические характеристики в ударной волне.

\section{1.1}



\section{1.2}


\subsection{1.2.1}\label{sec:Bg_rate}




\subsection{1.2.2}




\FloatBarrier
\section{Заключение}
 

Получены следующие результаты:
\begin{enumerate}
\item 
\item 
\end{enumerate}


\clearpage