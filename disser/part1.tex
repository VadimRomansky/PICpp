\chapter{Моделирование излучения FBOT}\label{FBOT}

Последние наблюдения открыли новый класс объектов - быстрые голубые оптические транзиенты (FBOT)\cite{Drout2014, Margutti2014, Coppejans2020, Ho2020}. Вместе со слабыми гамма-всплесками они возможно принадлежат к переходному классу объектов между нерелятивистскими сверхновыми и обычными гамма-всплесками. Как показано в работе Шевалье и Ирвина \cite{Chevalier2011} наблюдаемые характеристики вспышки определяются длительностью действия центрального источника энергии выброса и его взаимодействием со внешними слоями взрывающейся звезды. При различных параметрах вспышка будет относиться к разным классам объектов. Известные на данный момент быстрые голубые оптические транзиенты AT2018cow \cite{Margutti2014}, CSS161010 \cite{Coppejans2020}, ZTF2018abvkwla \cite{Ho2020} and AT2020xnd \cite{Ho2021, Bright2021} характеризуются большой светимостью $L > 10^43 эрг/с$, коротким характерным временным масштабом порядка нескольких дней, низкой выброшенной массой эжекты и высокими скоростями.
Рассматриваемы в данной главе объект CSS161010, расположенный в карликовой галактике на расстоянии 150 мегапарсек имеет по оценкам следующие\cite{Coppejans2020} характеристики: скорость эжекты 0.55 c на ранних этапах и около 0.3 c на поздних (порядка одного года), выброшенная масса порядка 0.01-0.1 солнечных масс. Важнейшей особенностью этих объектов является вид их спектральной плотности излучения, имеющей выраженный максимум и степенные участки, говорящие о наличии синхротронного самопоглощения.
%почему четвертый был отдельно в статье?


В данной главе будет проведено Particle-in-Cell моделирование ускорения частиц в транс-релятивистской ударной волне со скоростями, характерными для FBOT объектов. По имеющимся распределениям будет расчитано синхротронное излучение и с помощью фитирования наблюдательных данных будут подобраны такие параметры как магнитное поле, концентрация вещества и геометрические характеристики в ударной волне.

\section{Излучение объектов с синхротронным самопоглощением}
Процесс синхротронного излучения хороши известен и описан в классических работах. Но с точки зрения квантовой электродинамки, любому процессу излучения можно так же сопоставить процесс поглощения. Сечение процесса синхротронного самопоглощения описано в работе Гизеллини и Свенсона \cite{Ghisellini1991}. Спектральная плотность мощности излучения единицы объема вещества определеяется формулой
\begin{equation} \label{emission}
I(\nu)=\int_{E_{min}}^{E_{max}} dE \frac {\sqrt {3}{e}^{3}n F(E) B \sin ( \phi)}{{m_e}{c}^{2}}
\frac{\nu}{\nu_c}\int_{\frac {\nu}{\nu_c}}^{\infty }\it K_{5/3}(x)dx,
\end{equation}
где $\phi$ это угол межде вектором магнитного поля и лучом зрения, $\displaystyle\nu_{c}$ критическая частота, определяемая выражением $\displaystyle\nu_{c} = 3 e^{2} B \sin(\phi) E^{2}/4\pi {m_{e}}^{3} c^{5}$, и~$K_{5/3}$ - функция МакДональда.
Коэффициент поглощения для фотонов, распростроняющихся вдоль луча зрения равен
\begin{equation}\label{absorption}
k(\nu)=\int_{E_{min}}^{E_{max}}dE\frac {\sqrt {3}{e}^{3}}{8\pi m_e \nu^2}\frac{n B\sin(\phi)}{E^2}
\frac{d}{dE} E^2 F(E)\frac {\nu}{ \nu_c}\int_{\frac {\nu}{ \nu_c}}^{\infty }K_{5/3}(x) dx.
\end{equation}
Используя эти формулы Шевалье \cite{Chevalier1998} построил модель излучения плоского однородного диска, расположенного перпендикулярно к лучу зрения, с радиусом $R$, толщиной $s$, однородным магнитным полем $B$ и степенной функцией распределения электронов $N(E) = N_0 E^{-p}$. Вместо толщины обычно используется доля излучающего объема $f$, определенная так, что $\pi R^2 s = 4 \pi /3 R^3 f$. При таких предположениях \textcolor{red}{спектральная плотность излучения определяется следующими формулами}
\begin{equation}
I_{\nu}=S(\nu_1)J(\frac{\nu}{\nu_1},p)
sin(\theta)}^{\frac{p+2}{p+4}},
\end{equation}
where
\begin{equation}
\nu_1 = 2 c_1 {(s c_6 N_0)}^{\frac{2}{p+4}}{B sin(\theta)}^{\frac{p+2}{p+4}}
\end{equation}
\begin{equation}
S(\nu_1)=\frac{c_5}{c_6}{(B sin(\theta))}^{-1/2}{(\frac{\nu_1}{2 c_1})}^{5/2}
\end{equation}
\begin{equation}
J(z,p)=z^{5/2}(1-exp(-z^{(p+4)/2}))
\end{equation}
\section{Particle-in-Cell моделирование }


\section{Расчёт излучения}




\section{Результаты}




\FloatBarrier
\section{Заключение}
 

Получены следующие результаты:
\begin{enumerate}
\item 
\item 
\end{enumerate}


\clearpage