\chapter{Поиск гигантских вспышек от мягких гамма-репитеров в ближайших галактиках 
         среди коротких всплесков Конус-Винд} \label{SGR_GF_search}

\section{Введение}
Мягкие гамма-репитеры (SGRs) относятся к редкому классу нейтронных звёзд, проявляющих 
два типа активности в жестком рентгеновском диапазоне ($\sim 10\textrm{--}1000$~кэВ). 
Во время периода активности SGRs испускают короткие ($\sim0.001\textrm{--}1$~c) жесткие рентгеновские всплески 
с пиковой светимостью $10^{38}\textrm{--}10^{42}$~эрг~с$^{-1}$. Фаза активности может длиться 
от дней до года, после чего наступает длительная фаза затишья. Значительно реже, 
возможно, один раз за время нахождения нейтронной звезды в стадии SGR, SGR может 
производить гигантские вспышки (GF), во время которых высвобождается значительная 
энергия $\sim(0.01\textrm{--}1)\times 10^{46}$~эрг. Гигантская вспышка начинается 
с короткого ($\sim 0.2\textrm{--}0.5$~с) жесткого импульса с быстрым нарастанием, 
порядка нескольких миллисекунд, и более медленным спаданием, который переходит 
в длинный затухающий хвост, модулированный вращением нейтронной звезды. 
Пиковая светимость начального импульса может достигать $\sim 10^{47}$~эрг~с$^{-1}$.
Подробное описание наблюдательных свойств SGRs дано в работе~\citep{Mereghetti2013}.

На конец 2015~г. известно 15 SGR~\citep{Olausen_Kaspi2014}, из которых 14 
находятся в нашей Галактике и один расположен в Большом Магеллановом Облаке. 

Первая гигантская вспышка была зарегистрирована от SGR в Большом Магеллановом Облаке 
5~марта 1979~года приборами Конус на советских межпланетных станциях 
<<Венера-11 и -12>>~\citep{Golenetskii1979SvAL, Mazets1979} 
и сетью IPN~\citep{Barat1979, Cline1980, Evans1980, Cline1982}. 
На конец 2015~г. гигантские вспышки наблюдались только у трёх источников 
SGR~0526$-$66, SGR~1900$+$14 и SGR~1806$-$20, все они были зарегистрированы 
приборами Конус и сетью IPN. Две недавние вспышки от SGR~1900$+$14 и~SGR~1806$-$20 
сопровождались возрастанием вспышечной активности~\citep{Mazets1999a, Frederiks2007}.

Все известные SGR являются быстро замедляющимися рентгеновскими пульсарами 
с периодами 2--12~с и рентгеновскими светимостями в спокойном состоянии 
$(0.1\textrm{--}1)\times 10^{36}$~эрг~с$^{-1}$. Считается, что SGR принадлежат к 
более широкому классу магнетаров. Этот класс также включает аномальные 
рентгеновские пульсары (AXP) и радиопульсары с сильным магнитным полем (high-B radio pulsars).
Граница между SGR и AXP размыта из-за наличия аномально высокой рентгеновской светимости SGR,
не объясняемой в рамках существующих моделей остывания нейтронных звёзд, 
и регистрации жестких рентгеновских вспышек от AXPs. Примерно половина известных 
магнетаров отождествлена с областями активного звездообразования или остатками сверхновых.
Пространственное распределение магнетаров в Галактике схоже с распределением 
наиболее массивных звёзд класса O~\citep{Olausen_Kaspi2014}. Считается, что 
активность магнетаров связана с наличием у них сверхсильного магнитного поля,
оцененного из высокой скорости замедления SGR, 
$\sim 10^{13}\textrm{--}10^{15}$~Гс,~\citep{Duncan_and_Thompson_1992ApJ,Thompson_and_Duncan_1995MNRAS,Thompson_and_Duncan_1996ApJ}.
Альтернативные модели, описывающие активность SGR и AXP, приведены в работе~\citep{Bisnovatyi-Kogan_2014ARep}.

Благодаря огромной светимости начального импульса гигантские вспышки возможно  
регистрировать от SGR, расположенных в ближайших галактиках. При этом начальный 
импульс будет неотличим от короткого гамма-всплеска. Идея о возможности наблюдения 
гигантских вспышек в ближайших галактиках впервые была высказана 
в работах~\citep{Mazets1981,Mazets1982}. Оценки частоты гигантских вспышек от одного 
SGR и их доля среди коротких всплесков важны для понимания механизма генерации 
гигантских вспышек в рамках модели магнетара.

К настоящему времени обнаружено четыре коротких гамма-всплеска, которые могут 
являться GF в ближайших галактиках. У всплеска GRB~970110~\citep{Crider2006} 
по данным BATSE после начального импульса с длительностью $\approx 0.4$~с была 
обнаружена пульсирующая компонента с периодом 13.8~с на интервале 100~с. В область 
локализации этого всплеска попадает только одна близкая галактика NGC~6946 
на расстоянии 5.9~Мпк. 

В работе~\citep{Levan2008} обсуждается возможность того, что всплеск GRB~050906, 
зарегистрированный \textit{Swift}, является GF в галактике 
IC~328 на расстоянии 138~Мпк. Свидетельством в пользу этой гипотезы авторы считают  
отсутствие детектирования спадающего рентгеновского и оптического послесвечений этого всплеска, 
однако, это не кажется удивительным, так как всплеск является наиболее слабым всплеском 
в каталоге \textit{Swift}-BAT~\citep{Sakamoto2011ApJS}. При этом 
авторами не отвергается возможность принадлежности источника всплеска к скоплению 
галактик на $z = 0.43$. 

Всплеск GRB~051103, зарегистрированный сетью IPN, считается 
кандидатом в GF из группы галактик M81/M82, находящийся на расстоянии 
3.6~Мпк~\citep{Ofek_2006ApJ, Frederiks_2007AstLett, Hurley2010}. 
Этот всплеск представляет собой короткий импульс длительностью 170~мс с быстрым 
нарастанием ($<6$~мс) и экспоненциальным спадом с постоянной времени $\approx 55$~мс. 
Спектр наиболее интенсивной части всплеска описывается моделью 
CPL с $\alpha \approx 0.1$ и $E_\rmn{p} \approx 2.4$~МэВ. Локализация всплеска 
покрывает большое число рентгеновских источников в группе галактик M81/M82. Однако, 
в работе~\citep{Hurley2010} приводятся доводы против гипотезы о GF, в основном 
на основании гигантской пиковой светимости вспышки $\approx 4.7\times10^{48}$~эрг~с$^{-1}$,
если предположить, что источник находился в в группе галактик M81/M82.
Эта величина на порядок больше пиковой светимости вспышки от SGR~1806$-$20, которая 
составляла $(2\textrm{--}5)\times 10^{47}$~эрг~с$^{-1}$ в предположении 
расстояния до SGR 15~кпк.

Другой всплеск GRB~070201~\citep{Mazets_2008ApJ,Ofek_2008ApJ} имеет время нарастания $\approx 25$~мс 
и длительность $\approx 180$~мс. Спектр всплеска хорошо описывается степенным законом 
с экспоненциальным завалом~(формула~\ref{eq:CPL}) с $\alpha\approx-0.6$ и $E_\rmn{p}\approx 280$~кэВ. 
Область локализации этого всплеска накладывается на галактику M31 (0.78~Мпк). У всплеска обнаружено 
мягкое послесвечение в диапазоне 17--70~кэВ на интервале до 94~с после начального 
импульса, которое было интерпретировано как хвост GF.

Пределы на долю GF среди коротких гамма-всплесков были получены в нескольких 
работах~\citep{Lazzati2005,Palmer2005,Nakar2006ApJ,Popov2006,Ofek_2007ApJ,Tikhomirova2010AstL} 
и составили 1--15\%, см.~работу~\citep{Hurley2011} с подробным обзором предыдущих результатов.

Каталог локализаций коротких гамма-всплесков KW~\citep{Palshin2013}, 
описанный в предыдущей главе, содержит информацию о локализации 271 короткого 
всплеска, зарегистрированного за 16~лет практически непрерывных наблюдений всей 
небесной сферы. Этот каталог позволил осуществить поиск кандидатов в GF 
в наибольшем на 2014~г. наборе точно локализованных коротких всплесков.

В разделе~\ref{KW_sensitivity} обсуждается чувствительность KW и IPN к GF.
В разделе~\ref{Gal_sample} приводится набор близких галактик и обсуждаются их свойства. 
В разделе~\ref{GF_search} описывается поиск близких галактик в областях локализации 
коротких всплесков. На основании результатов этого поиска в разделе~\ref{GF_rate} 
вычисляется верхний предел на частоту гигантских вспышек от SGRs. Заключительные 
ремарки приведены в разделе~\ref{Summary}.

\section{Чувствительность KW и IPN}\label{KW_sensitivity}
Начальный импульс гигантской вспышки из близкой галактики вызовет срабатывание 
триггера на временном масштабе 140~мс. Среди трёх зарегистрированных гигантских 
вспышек ни одна не имеет достоверных прямых измерений параметров начального импульса 
из-за экстремальных потоков падающего излучения, перегружающих измерительные 
тракты всех гамма-детекторов. Значение потока, приводящего к насыщению детекторов 
KW составляет примерно $2.4 \times 10^{-2}$~эрг~см$^{-2}$~с$^{-1}$~\citep{Mazets1999a}.

Параметры начального импульса GF от SGR~1806$-$20 были определены благодаря 
регистрации излучения вспышки, отраженного от Луны~\citep{Frederiks2007}. 
Спектр излучения хорошо описывается моделью CPL с параметрами
$\alpha=-0.73_{-0.64}^{+0.47}$ и $E_{0}=666^{+1859}_{-368}$~кэВ 
(что соответствует энергии максимума $\nu F_{\nu}$ спектра~--- 
$E_{\rmn{p}} = (2-\alpha)E_0 = 850^{+1259}_{-303}$~кэВ). В работе~\citep{Terasawa2005} 
был восстановлен только временной профиль этой гигантской вспышки. 
Для двух других вспышек удалось получить только грубые оценки на $E_{\rmn{p}}$: 
$\sim 400\textrm{--}500$~кэВ (SGR~0526$-$66,~\citep{Golenetskii1979SvAL, Mazets1979}) 
и $>250$~кэВ (SGR~1900+14,~\citep{Hurley1999, Mazets1999}).

В предположении, что вся энергия начального импульса выделяется на коротком 
триггерном масштабе (140~мс), был оценен минимальный интегральный поток $S_\rmn{min}$ 
в диапазоне 20~кэВ--10~МэВ, который даст превышение скорости счёта над фоном на $9 \sigma$ 
в энергетическом диапазоне G2 (50--200~кэВ) при скорости счёта фона 400~отсч~с$^{-1}$. 
На основании полученного значения $S_{\rmn{min}}$, предельное расстояние регистрации 
вычислялось по формуле $d_{\rmn{max}} = \sqrt{Q/(4\pi S_{\rmn{min}})}$, где 
$Q$~[эрг]~--- энерговыделение начального импульса.

Минимальный поток $S_\rmn{min}$ сильно зависит от жесткости всплеска ($\alpha$ 
и $E_{\rmn{p}}$). Известный диапазон $E_\rmn{p}$ начальных импульсов 
GF~--- 200~кэВ--1~МэВ соответствует диапазону 
$S_\rmn{min}= (2.1\textrm{--}5.7)\times 10^{-7}$~эрг~см$^{-2}$, 
см.~рис.~\ref{img:KW_lim_distance}. 

\begin{figure}[h] 
    \center
    \includegraphics [width=0.8\textwidth] {gLimDist18to30RU.eps}
    \caption[Зависимость минимального интегрального потока за 140~мс 
    (20~кэВ--10~МэВ) от пиковой энергии спектра.]
    {Зависимость минимального интегрального потока за 140~мс (20~кэВ--10~МэВ) 
и предельного расстояния, в предположении энерговыделения $Q=2.3\times 10^{46}$~эрг,
от параметров спектральной модели CPL $E_\rmn{p}$ и $\alpha$: сплошная линия~--- $\alpha=-1$, 
штриховая линия~--- $\alpha=-1.5$, пунктирная линия~--- $\alpha=-0.5$. 
Штрихованная область~--- диапазон предельных расстояний $\approx 18\textrm{--}30$~Мпк 
соответствующий диапазону $E_\rmn{p} = 250$~кэВ--1~МэВ.}
 \label{img:KW_lim_distance}
\end{figure}

Также была исследована зависимость площадей областей локализаций IPN коротких   %<--скольки областей?
всплесков от потока, измеренного KW в диапазоне 20~кэВ--10~МэВ на интервале 140~мс 
с наибольшей скоростью счёта, при этом не было обнаружено существенной корреляции, 
см.~рис.~\ref{img:IPN_box_area}. Таким образом, можно использовать полученный интервал 
значений $S_\rmn{min}$ для всей IPN.

\begin{figure}[h]
    \center
    \includegraphics [width=0.8\textwidth] {gAreaVsFluenceRU.eps}
    \caption{Площадь области локализации IPN в зависимости от интегрального потока 
    (20~кэВ--10~МэВ) на интервале 140~мс с максимальной скоростью счёта.}
    \label{img:IPN_box_area}
\end{figure}

Оценки расстояния до SGR~1806$-$20 находятся в диапазоне примерно от 6~кпк 
до~19~кпк~\citep{Tendulkar2012ApJ}, по самым последним оценкам~\citep{Svirski2011} 
диапазон расстояний составляет 9.4--18.6~кпк.

Изотропное энерговыделение начального импульса GF от SGR~1806$-$20 
($Q = 2.3\times 10^{46} d_{15}^2 $~эрг) предполагает предельное расстояние 
детектирования $d_\rmn{max} = (18\textrm{--}30)\times d_{15}$~Мпк, 
где $d_{15}=d/15$~кпк и $d$~--- расстояние до SGR~1806$-$20. 
Менее интенсивные GF с энерговыделением $Q \approx 10^{45}$~эрг 
(сравнимые со вспышкой 5 марта 1979~г. от SGR~0526$-$66) 
могут быть зарегистрированы до расстояний $d_\rmn{max} = (3.8\textrm{--}6.3) Q_{45}^{0.5}$~Мпк, 
где $Q_{45}$~--- энерговыделение вспышки в единицах $10^{45}$~эрг. 
Диапазон значений $d_\rmn{max}$ является трансляцией диапазона $S_\rmn{min}$.

\section{Набор близких галактик}\label{Gal_sample}
Наиболее полным каталогом, содержащим параметры близких галактик является 
каталог для поиска источников гравитационных волн (Gravitational Wave Galaxy Catalogue, 
GWGC~\citep{White2011CQGra}). Каталог содержит более 53000 галактик на расстояниях 
до~100~Мпк. Для поиска возможных родительских галактик гигантских вспышек SGR 
изначально был выбран набор из 8112 галактик на расстояниях до~30~Мпк. 
Неопределённость расстояний до галактик в наборе составляет 15--22\%.

Используя метод предложенный в работе~\citep{Ofek_2007ApJ}, была оценена полнота набора, 
связанная с затенением Галактикой. Было построено распределение галактик по Галактической 
широте с шагом $10^\circ$. Разность числа галактик в интервале 
$0^\circ\textrm{--}10^\circ$ и числа галактик в незатенённом интервале 
$10^\circ\textrm{--}20^\circ$, отнесённая к общему числу галактик, даёт долю 
потерянных галактик 6\% и полноту набора галактик $\epsilon_{\rmn{G}}=94$\%.

В предположении, что все SGR~--- молодые изолированные нейтронные звёзды, считалось, 
что число SGR пропорционально частоте вспышек сверхновых с коллапсом ядра 
(CCSN, типы Ib/c и~II) в галактике.

Следуя работам~\citep{Cappellaro1999, Boser2013} считалось, что частота вспышек 
сверхновых пропорциональна светимости галактики в фильтре 
$B$\footnote{Параметры фильтра $B$: $\lambda=445$~нм и $\rmn{FWHM}=94$~нм~\citep{Binney1998GalAstr}} 
($L_{B}$) $R_{\rmn{SN}} = k L_{B}$, 
где $k$ множитель зависящий от морфологического типа галактики данный 
в единицах SNu, см.~таб.~\ref{tab:RateCCSN}.

\begin{table} [h]
  \centering
  \scriptsize
  \parbox{15cm}{\caption{Ожидаемая частота вспышек CCSN в зависимости от типа галактики}
  \label{tab:RateCCSN}}
  \begin{tabular}{ccc}
  \hline
  \hline
  Тип галактики & Числовой тип &  Частота вспышек CCSN ($k$) в SNu \\
  по Хабблу     &  по Хабблу   &           \\     
  \hline
   E-S0           & от $-6$ до $-1$ &  $<0.05$ \\
   S0a-Sb         &  $0$--$3$   &  $0.89\pm0.33$ \\
   Sbc-Sd         &  $4$--$7$   &  $1.68\pm0.60$ \\
   Sm, Irr., Pec. &  $8$--$10$  &  $1.46\pm0.71$\\
  \hline
  \end{tabular}
\end{table}

Единица SNu соответствует $1\rmn{SN}(100\rmn{год})^{-1}(10^{10}L_{\odot B})^{-1}$, 
где $L_{\odot B} = 2.16\times10^{33}$~эрг~с$^{-1}$ светимость Солнца в фильтре~$B$. 
Светимость галактики $L_{B}$ вычислялась по формуле 
$L_{B}=10^{-0.4(M_B-M_{\odot B})} L_{\odot B}$, 
где $M_{\odot B}=5.48$ абсолютная звёздная величина Солнца в фильтре~$B$. 

Исходный набор 8112 галактик содержит 790 галактик, для которых не дана $L_{B}$,
таким образом полнота набора по $L_{B}$ составляет $\epsilon_{L} \approx 90$\%. 
Среди оставшихся 7322 галактик с указанным $L_{B}$, для 2405 не указан морфологический тип. 
Яркость этих галактик в среднем меньше на три звёздные величины по сравнению 
с классифицированными галактиками, при этом они содержат меньше 7\% суммарной 
частоты вспышек сверхновых, поэтому эти галактики были исключены из дальнейшего 
рассмотрения. 

В качестве итогового набора галактик был взяты 1896 галактики поздних типов 
(все кроме E и S0) с наибольшими $R_{\rmn{SN}}$, содержащие $\epsilon_{\rmn{SN}}=90$\% 
от суммарной частоты вспышек сверхновых. Плотность этих галактик на небесной сфере 
составляет 0.046~градус$^{-2}$. Суммарная частота вспышек сверхновых составляет 
$R_\rmn{SN}=22.8 \pm 0.4$~год$^{-1}$.

Для проверки методики определения $R_{\rmn{SN}}$, объемная плотность частоты 
вспышек сверхновых $R_\rmn{SN}(d)/(4/3\pi d^3)$, где $d$~--- расстояние, 
для набора 1896 галактик сравнивалась 
с нижним пределом $1.9_{-0.2}^{+0.4}\times 10^{-4}$~год$^{-1}$~Мпк$^{-3}$, 
полученным в обзоре сверхновых~\citep{Mattila2012} в галактиках ближе 15~Мпк. 
Зависимость плотности $R_{\rmn{SN}}$ от расстояния представлена на рис.~\ref{img:RateCCNvsDist}. 
Объёмная плотность, полученная на основе голубой светимости галактик согласуется в 
пределах $1 \sigma$ с наблюдаемой величиной, предполагая, что $\approx 19$\% близких CCSN 
были пропущены оптическими обзорами неба.

Объёмная плотность частоты вспышек CCSN демонстрирует значимый спад на расстояниях 
свыше $\approx 22$~Мпк, что может быть связано с падением полноты набора галактик 
на б\'{о}льших расстояниях. Для оценки объёмной частоты вспышек CCNS на расстояниях 
больших 22~Мпк было использовано среднее значение в диапазоне расстояний 
до 22~Мпк~--- $(2.74 \pm 0.18) \times 10^{-4}$~год$^{-1}$~Мпк$^{-3}$, 
ошибка дана на уровне значимости $1\sigma$.

Так же было обнаружено увеличение $R_{\rmn{SN}}/V$ внутри $\sim 10$~Мпк вплоть до 
$(9.3 \pm 0.16) \times 10^{-4}$~год$^{-1}$~Мпк$^{-3}$ внутри объёма 5~Мпк. 
При этом всего пять галактик содержат 25\% общей частоты CCSN:
PGC047885 на расстоянии $d = 5$~Мпк, IC~0342 на расстоянии 3.28~Мпк, NGC~6946 на 
расстоянии 5.9~Мпк, NGC~5457~(M101) на расстоянии 6.7~Мпк и  NGC~5194~(M51) 
на расстоянии 5.9~Мпк. Эти галактики являются наиболее вероятными  источниками 
гигантских вспышек SGR в ближайшей Вселенной в дополнение к предложенным ранее 
в работе~\citep{Popov2006}: M82 на расстоянии $d = 3.4$~Мпк, NGC~253 на расстоянии 2.5~Мпк, 
NGC~4945 на расстоянии 3.7~Мпк и M83 на расстоянии 3.7~Мпк. 

\begin{figure}[h]
    \center
    \includegraphics [width=0.8\textwidth] {gRsn2VpubRU.eps}
    \caption[Удельная частота вспышек CCSN в зависимости от расстояния]
	{Удельная частота вспышек CCSN в зависимости от расстояния $d$. 
	Сплошная линия~--- $R_{\rmn{SN}}/V(d)$ для 1896 галактик, 
	содержащих 95\% $R_{\rmn{SN}}$ внутри 30~Мпк, из каталога GWGC. 
	Горизонтальные штриховые линии обозначают $\pm 1\sigma$ интервал для 
	локальной удельной частоты вспышек сверхновых 
	$2.3_{-0.2}^{+0.5}\times 10^{-4}$~год$^{-1}$~Мпк$^{-3}$ из работы~\citep{Mattila2012}, 
    предполагая, что доля пропущенных в обзоре сверхновых составляет 0.189.}
    \label{img:RateCCNvsDist}
\end{figure}

\section{Поиск гигантских вспышек среди коротких гамма-всплесков, 
зарегистрированных Конус-Винд}\label{GF_search}

Каталог локализаций коротких всплесков Конус-Винд, полученных при помощи 
сети IPN~\citep{Palshin2013}, содержит 271 всплеск, зарегистрированный 
по крайней мере одним КА IPN, что дало возможность локализовать их при помощи 
триангуляции (см.~Главу~\ref{IPN_catalog}). Этот набор содержит 30 всплесков, 
классифицированных как короткие гамма-всплески с продлённым излучением.

Процедура поиска наложений галактик и локализаций всплесков была проведена 
для нескольких поднаборов всплесков.  При поиске наложений галактика моделировалась 
кругом с центром, взятым из каталога, и диаметром, равным большой полуоси галактики. 
Оценка ожидаемого числа галактик в этом наборе областей локализации была выполнена 
методом Монте-Карло. Было сгенерировано 1000 реализаций набора галактик, в которых 
центры галактик выбирались случайным образом. Для каждого набора вычислялось число 
наложений галактик на локализации. Полученные числа сортировались по возрастанию 
и в качестве границ 95\% доверительного интервала для числа наложений 
бралось 25-е и~975-е число.

Наибольший набор включал 140 всплесков с площадями областей локализации 
меньше 10~кв.~градусов.
Было обнаружено, что на 12 из 140 IPN локализаций с общей площадью 217~кв.~градусов 
накладывается 20 галактик, при этом ни одна локализация всплеска с продлённым 
излучением не содержит галактики. Для этого набора локализаций число галактик, 
ожидаемое для случайного наложения, равно 22--44 на уровне значимости 95\%. 
Также было обнаружено, что локализация только одного всплеска (GRB~050312) 
накладывается на окраину скопления Девы, при этом локализация не накладывается 
ни на одну галактику из набора. Здесь скопление девы моделировалось кругом с центром в 
$\rmn{R.A.}=188^\circ$, $\rmn{Dec.}=12^\circ$ с радиусом $6^\circ$, 
параметры скопления были взяты из работы~\citep{Binggeli1987}.

Поиск наложений показал, что только два всплеска имеют малую вероятность случайного 
наложения на близкую галактику $P_\rmn{chance} \sim 1$\%. Эти всплески ранее были 
отнесены к внегалактическими GF: GRB~051103, чья локализация накладывается на группу галактик M81/M82 
(площадь бокса $4.3\times10^{-3}$~кв.~градусов) и GRB~070201, считающийся GF в галактике 
Андромеды (площадь бокса 0.123~кв.~градусов). Вероятность $P_{\rmn{chance}}$ 
соответствует обнаружению как минимум одной галактики в заданной локализации (боксе).

Затем процедура поиска была применена к набору 98 локализаций с площадью меньше 
1~кв.~градус, который содержит всплески, зарегистрированные по крайней мере одним 
удалённым космическим аппаратом~(см. главу~\ref{IPN_catalog}). Было обнаружено, 
что только локализации двух упомянутых выше всплесков содержат галактики.

Общей чертой всех известных GF является малая длительность начального импульса 
$\lesssim 500$~мс и малое время нарастания импульса $t_{\rmn{r}} \lesssim 25$~мс. 
Среди 296 коротких всплесков 40 имеют $t_{\rmn{r}}<25$~мс и длительность $<500$~мс. 
Ранее обнаруженные кандидаты GRB~051103 и GRB~070201 имеют $t_{\rmn{r}}=2$~мс 
и~$t_{\rmn{r}}=24$~мс соответственно. Среди коротких всплесков KW этим 
критериям удовлетворяют 17 событий с площадями областей локализации менее 10~кв.~градусов.
Было обнаружено, что четыре области локализации с общей площадью 47~кв.~градусов накладываются на пять галактик. 
Это число попадает в 95\% доверительный интервал для случайного наложения, 5--16 галактик. Из этих 
четырёх всплесков только GRB~051103 и GRB~070201 имеют малые вероятности случайного наложения.
Результаты поиска наложений для всех перечисленных наборов всплесков приведены 
в Таблице~\ref{tab:SearchResults}.

\begin{table}[h]
  \centering
  \scriptsize
  \caption{Результаты поиска кандидатов в гигантские вспышки SGR в 
  наборе коротких всплесков Конус-Винд}
  \label{tab:SearchResults}
  \begin{tabular}{cccc}
  \hline
  \hline
Описание & Число локализаций  & Число галактик & Ожидаемое число  \\
набора   &                    & в локализациях & наложений на уровне 95\% \\
\hline
Площадь локализации $<10$~кв.~градусов  & 140 & 20 & 22--44 \\
Площадь локализации $<1$~кв.~градусов   & 98 & 2 & 0--7\\
$t_{\rmn{r}} \leq 25$~мс, $T_{100}<500$~мс, & 17 & 5 & 5--16\\
и площадь локализации $<10$~кв.~градусов & & & \\
\hline
\end{tabular}
\end{table}

С учетом произведения факторов полноты набора галактик $\epsilon_{\rmn{G}} \epsilon_{\rmn{L}} \epsilon_{\rmn{SN}} \approx 76$\%, 
и предполагая, что было найдено два кандидата в GF среди 98 хорошо локализованных коротких всплесков, 
можно поставить верхний предел на долю GF среди коротких всплесков Конус-Винд 
$<8$\% (=6.296/98/0.76), где 6.296~--- 95\% односторонний верхний предел на 
число вспышек~\citep{Gehrels1986}. Благодаря непрерывному наблюдению всего неба IPN, 
этот предел может быть распространён на всю популяцию коротких гамма-всплесков с 
интегральными потоками выше $\sim 5\times 10^{-7}$~эрг~см$^{-2}$. Полученный верхний 
предел жестче чем полученный в работе~\citep{Ofek_2007ApJ}.

\section{Верхний предел на частоту гигантских вспышек}\label{GF_rate}
Предполагая, что только одна GF с энерговыделением $Q \gtrsim 10^{46}$~эрг наблюдалась 
в группе галактик M81/M82 внутри объёма $d \le 30$~Мпк, можно получить верхний предел 
на частоту подобных GF. Оценка была основана на предположении, что число активных SGR ($N_\rmn{SGR}(d)$) 
внутри сферы радиуса $d$ пропорционально частоте вспышек CCSN 
$R_\rmn{SN}(d)=4/3 \pi d^3 r_\rmn{SN}$, где $r_\rmn{SN}$~--- объёмная частота вспышек CCSN.
\begin{equation}\label{eq:NumSGR}
N_{\rmn{SGR}} (d) = \frac{N_{\rmn{SGR}, \rmn{MW+LMC}}}{ R_{\rmn{SN}, \rmn{MW+LMC}}} R_{\rmn{SN}}(d).
\end{equation}
Галактическая частота CCSN равна $R_{\rmn{SN}, \rmn{MW}} = 0.028\pm0.006$~в~год 
с систематической ошибкой $\sim 2$~раза~\citep{Li2011part3}, и частотой в Большом 
Магеллановом облаке (LMC)~$R_{\rmn{SN}, \rmn{LMC}} = 0.013\pm0.009$~в~год~\citep{Bergh1991}. 
Таким образом, суммарная частота равна $R_{\rmn{SGR}, \rmn{MW+LMC}} = 0.041\pm0.011$~в~год. 

% Значение $R_{\rmn{SN}, \rmn{MW}} = 0.028\pm0.006$~в~год получено в Li et al., 2011
% экстраполяции R_SN полученной для галактик схожих с MW. Тип MW -Sbc 
% (спиральная с балджем и "неплотной намоткой рукавов").
% На самом деле R_SN в LMC ~1 в 1000 лет (0.001 yr-1) а не 0.013+/-0.009 yr-1, 
% определена по наблюдениям остатков сверхновых!!!

Наблюдаемая частота гигантских вспышек на SGR задаётся выражением
\begin{equation}\label{eq:RateGF}
R_{\rmn{GF}} = \frac{N_{\rmn{GF},\rmn{obs}}}{\Delta T N_{\rmn{SGR}}(d_{\rmn{max}})} ,
\end{equation}
где $N_{\rmn{GF},\rmn{obs}}$~--- число зарегистрированных GF, 
$\Delta T=16$~лет~--- время наблюдения KW на 2010~г. и $N_{\rmn{SGR}}(d_{\rmn{max}})$ 
задано уравнением~\ref{eq:NumSGR}. Для оценки верхнего предела на $R_{\rmn{GF}}$ 
в случае одной зарегистрированной GF использовался 95\% односторонний верхний предел
на $N_{\rmn{GF}, \rmn{obs}}=4.744$~\citep{Gehrels1986}.

Для частоты GF с энерговыделением $Q \gtrsim 10^{46}$~эрг в объеме $d\le30$~Мпк уравнение~\ref{eq:RateGF} 
даёт верхний предел ${(0.6\textrm{--}1.2)\times 10^{-4} Q_{46}^{-1.5}}$~год$^{-1}$~SGR$^{-1}$, 
где $Q_{46}$ энерговыделение GF в единицах $10^{46}$~эрг. Диапазон верхнего предела 
является трансляцией ошибок частоты вспышек CCSN. Регистрация только 
одной GF с энерговыделением $Q \gtrsim 10^{46}$~эрг за последние 35~лет 
с 1979~г от SGR~1806$-$20 предполагает частоту таких вспышек в галактике 
${(0.005\textrm{--}1)\times 10^{-2}}$~год$^{-1}$~SGR$^{-1}$ ($=1_{-0.98}^{+4.6} / 35 / 15 $) 
для одностороннего 95\% уровня значимости. Это величина согласуется с верхним пределом, 
вычисленным выше с учётом диапазона расстояний до SGR~1806$-$20 9.4--18.6~кпк, 
в основном из за больших неопределённостей в галактической частоте GF. 

Для менее интенсивных GF с энерговыделением $Q \lesssim 10^{45}$~эрг, которые 
могут регистрироватся KW и IPN на расстояниях до 6.3~Мпк, считая что 
одна такая вспышка была зарегистрирована из галактики Андромеды, верхний предел 
на частоту вспышек составляет
${(0.9\textrm{--}1.7)\times 10^{-3}}$~год$^{-1}$~SGR$^{-1}$. Этот предел согласуется
с наблюдаемой галактической частотой таких вспышек 
${(0.5\textrm{--}1.4)\times 10^{-2}}$~год$^{-1}$~SGR$^{-1}$. Верхний предел и 
интервал галактической частоты GF дан на уровне значимости 95\%.

Дополнительно был вычислен верхний предел на частоту ярких GF с использованием 
данных \textit{Swift}-BAT. С момента запуска в ноябре 2004~г. \textit{Swift} наблюдал 
только один кандидат в GF, GRB~050906~\citep{Levan2008}, предположительно из 
галактики IC~328 на расстоянии $\approx 130$~Мпк. Этот всплеск имел наименьший 
интегральный поток в диапазоне 15--150~кэВ из всех зарегистрированных на 2011~г. всплесков, 
$S_{\rmn{min}} = 6.1\times10^{-9}$~эрг~см$^{-2}$~\citep{Sakamoto2011ApJS} и 
мягкий спектр, описываемый степенным законом с показателем~$-1.7$. Экстраполяция 
энерговыделения GF от SGR~1806$-$20 из диапазона 10~кэВ--10~МэВ в 15--150~кэВ, 
используя степенной закон с экспоненциальным завалом с параметрами $\alpha=-0.73$ 
и $E_{\rmn{p}}=850$~кэВ даёт $2.5\times10^{45}$~эрг, что соответствует предельному 
расстоянию детектирования 60~Мпк. Полученное предельное расстояние детектирования 
в совокупности с мягким спектром всплеска делает ассоциацию с SGR в IC~328 маловероятной.

Время наблюдения 90\% неба BAT в 2004--2010~гг. 
составило $7.25\times10^{6}$~с (0.23~года)~\citep{Baumgartner2013ApJS}, и экстраполяция 
этой величины на 2004--2013~гг. даёт 0.35~лет. Отсутствие зарегистрированных GF в 
объеме $d \lesssim 60$~Мпк в течение наблюдений \textit{Swift}-BAT даёт верхний предел 
на частоту GF на уровне 95\% $\sim 6 \times 10^{-4} Q_{45}^{-1.5} $~год$^{-1}$~на~SGR, 
где $Q_{45}$~--- энерговыделение GF в единицах $10^{45}$~эрг в диапазоне 15--150~кэВ. 
Таким образом, не смотря на высокую чувствительность BAT, полученный предел 
менее жесткий, чем полученный по данным KW и IPN из-за меньшей экспозиции всего неба.

\section{Заключение}\label{Summary}
В данной главе были получены следующие результаты:
\begin{enumerate}
\item Оценена чувствительность KW и IPN, и получено 
предельное расстояние регистрации гигантских вспышек SGR схожих с GF от SGR~1806$-$20 
равное $(18\textrm{--}30) d_{15}$~Мпк. Показано, что менее интенсивные GF, сравнимые 
с GF от SGR~1900+14 и SGR~0526$-$66 могут быть зарегистрированы IPN в галактиках 
не далее $\approx 6$~Мпк.

\item Произведён поиск близких галактик, находящихся ближе 30~Мпк, в локализациях 
коротких гамма-всплесков KW. Были обнаружены только два всплеска, ранее 
ассоциированые с группой галактик M81/M82 (GRB~051103) и галактикой Андромеды (GRB~070201),
локализации которых имеют малую вероятность случайного наложения на эти галактики.
Дополнительный поиск всплесков из скопления Девы не выявил возможных кандидатов в GF.

\item Получен верхний предел на частоту GF с энегрговыделением $Q \gtrsim 10^{46}$~эрг равный
${(0.6\textrm{--}1.2)\times 10^{-4} Q_{46}^{-1.5}}$~год$^{-1}$~на~SGR, который предполагает 
появление около одной GF с таким энерговыделением за время активности SGR, $10^3\textrm{--}10^5$~лет. 
Этот предел был вычислен на основе наибольшего на 2014~г.  
набора коротких всплесков и согласуется с ранее полученной в работе~\citep{Ofek_2007ApJ} оценкой.

Для GF, сопоставимых по энерговыделению со вспышкой 5 марта~1979~г. ($Q \lesssim 10^{45}$~эрг), 
полученный верхний предел на порядок выше~--- $(0.9\textrm{--}1.7)\times 10^{-3}$~год$^{-1}$~SGR$^{-1}$. 
Что может быть интерпретировано, как возможность наблюдать более одной подобной GF за время жизни SGR.
Полученные верхние пределы содержат неопределённость в порядок величины, связанную с
неопределённостью галактической частоты вспышек CCSN, расстояния до SGR~1806$-$20 и
предельного расстояния детектирования IPN.
Оцененные дипольные  магнитные поля SGR~1900+14 и SGR~0526$-$66, равные
$5.6\times10^{14}$~Гс и $7\times10^{14}$~Гс, соответственно~\citep{Olausen_Kaspi2014}, 
по-видимому, являются достаточными для генерации десятка GF с энерговыделением $Q \sim 10^{45}$~эрг,
что согласуется с полученными оценками на частоту появления GF.

\item Определены галактики, которые являются наиболее вероятными источниками GF 
из-за наибольшего оцененного количества SGR в этих галактиках. Это галактики
PGC047885, IC~0342, NGC~6946, NGC~5457 и NGC~5194, в дополнении к предложенным 
в работе~\citep{Popov2006}.
\end{enumerate}

По материалам Главы~\ref{SGR_GF_search} на защиту выносится следующее положение:
\begin{itemize}
\item Результаты поиска гигантских вспышек от мягких гамма-репитеров 
    в близлежащих галактиках по данным в эксперимента Конус-Винд. 
\end{itemize}

Результаты, полученные в главе, отражены публикации\\
D.~S. Svinkin, K. Hurley, R.~L. Aptekar, S.~V.~Golenetskii, D.~D.~Frederiks  
A~search for giant flares from soft gamma-repeaters in nearby galaxies in the 
Konus-Wind short burst sample // Mon.~Not.~R.~Astron.~Soc. 2015. Vol.~447,~1. p.~1028

\clearpage