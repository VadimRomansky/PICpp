\chapter{Particle-in-Cell моделирование транс-релятивистских ударных волн}\label{PIC}
\section{Введение}
Particle-in-Cell - распространенный метод численного моделирования плазмы, заключающийся в согласованном решении уравнений Максвелла для электро-магнитного поля и уравнений движения для частиц плазмы. Этот подход бул предложен и применен Оскаром Бунеманом \cite{Buneman1980} и другими.

Метод Particle-in-Cell предполагает, что плазма бесстолкновительная, и одночастичная функция распределения $F(t,x,p)$ частиц подчиняется уравнению Власова
\begin{equation} \label{Vlasov}
	\frac{\partial F}{\partial t}+v\cdot \nabla_x F + q\left(E+v\times B\right) \nabla_p F = 0
\end{equation}

В ходе численного расчета функция распределения представляется как сумма супер-частиц

\begin{equation}
	F(t,x,p) = \sum_{p=1}^{N} \frac{\omega_p}{V_c}S(x-x_p(t))\delta(p-p_p(t))
\end{equation}

где $\omega_p$ - вес супер-частицы (т.е. скольки реальным частицам она соответствует), $x_p, p_p$ - её координаты и импульс, $V_c$ - объем ячейки, $S$ - функция формы частицы, $\delta$ - делта-функция Дирака. При такой дискретизации, движение каждой супер-частицы будет подчиняться уравнениям движения

\begin{align}
\begin{split}
	\frac{dx_p}{dt}=v_p\\
	\frac{dp_p}{dt}=\frac{q}{m}\left(E_p+\frac{v_p}{c}\times B_p\right)
\end{split}
\end{align}

Где поля, действующие на частицу вычисляются усреднением по объему

\begin{equation}
	E_p = \frac{1}{V_c}\int d^3 x S(x-x_p)E(x)
\end{equation}

Временная эволюция электрического и магнитного полей вычисляется с помощью численного решения уравнений Максвела, с учетом тока, создаваемого частицами. Сложность заключается в том, что если вычислять ток, усредняя скорость движения частиц в данной ячейке, не будет выполняться уравнение непрерывности для плотности заряда. В результате потребуется решение уравнения Пуассона для коррекции электрического поля. Чтобы избежать этой проблемы, в коде SMILEI используется метод декомпозиции тока, предложенный Езиркеповым \cite{Esirkepov}, позволяющий автоматически сохранять уравнение непрерывности.

%\begin{equation}
%	J(x) = \sum_s q_s \int d^3 p v(p) F_s(x,p) = \sum_s %\sum_p q_s v(p) \frac{\omega_p}{V_c} \int d^3 x S(x-x_p)
%\end{equation}

\FloatBarrier
\section{Выводы}
 

Получены следующие результаты:
\begin{enumerate}
\item 
\item 
\end{enumerate}


\clearpage