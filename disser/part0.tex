\chapter{Particle-in-Cell моделирование транс-релятивистских ударных волн}\label{PIC}
\section{Введение}
Particle-in-Cell - распространенный метод численного моделирования плазмы, заключающийся в согласованном решении уравнений Максвелла для электро-магнитного поля и уравнений движения для частиц плазмы. Этот подход бул предложен и применен Оскаром Бунеманом \cite{Buneman1980} и другими.

Метод Particle-in-Cell предполагает, что плазма бесстолкновительная, и одночастичная функция распределения $F(t,x,p)$ частиц подчиняется уравнению Власова
\begin{equation} \label{Vlasov}
	\frac{\partial F}{\partial t}+v\cdot \nabla_x F + q\left(E+v\times B\right) \nabla_p F = 0
\end{equation}

В ходе численного расчета функция распределения представляется как сумма супер-частиц

\begin{equation}
	F(t,x,p) = \sum_{p=1}^{N} \frac{\omega_p}{V_c}S(x-x_p(t))\delta(p-p_p(t))
\end{equation}

где $\omega_p$ - вес супер-частицы (т.е. скольки реальным частицам она соответствует), $x_p, p_p$ - её координаты и импульс, $V_c$ - объем ячейки, $S$ - функция формы частицы, $\delta$ - делта-функция Дирака.

\FloatBarrier
\section{Выводы}
 

Получены следующие результаты:
\begin{enumerate}
\item 
\item 
\end{enumerate}


\clearpage