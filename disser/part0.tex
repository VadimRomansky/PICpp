\chapter{Particle-in-Cell моделирование транс-релятивистских ударных волн}\label{PIC}
\section{Введение}
Particle-in-Cell - распространенный метод численного моделирования плазмы, заключающийся в согласованном решении уравнений Максвелла для электро-магнитного поля и уравнений движения для частиц плазмы. Этот подход бул предложен и применен Оскаром Бунеманом \cite{Buneman1980} и другими.

Метод Particle-in-Cell предполагает, что плазма бесстолкновительная, и одночастичная функция распределения $F(t,x,p)$ частиц подчиняется уравнению Власова
\begin{equation} \label{Vlasov}
	\frac{\partial F}{\partial t}+v\cdot \nabla_x F + q\left(E+v\times B\right) \nabla_p F = 0
\end{equation}

В ходе численного расчета функция распределения представляется как сумма супер-частиц

\begin{equation}
	F(t,x,p) = \sum_{p=1}^{N} \frac{\omega_p}{V_c}S(x-x_p(t))\delta(p-p_p(t))
\end{equation}

где $\omega_p$ - вес супер-частицы (т.е. скольки реальным частицам она соответствует), $x_p, p_p$ - её координаты и импульс, $V_c$ - объем ячейки, $S$ - функция формы частицы, $\delta$ - делта-функция Дирака. При такой дискретизации, движение каждой супер-частицы будет подчиняться уравнениям движения

\begin{align}
\begin{split}
	\frac{dx_p}{dt}=v_p\\
	\frac{dp_p}{dt}=\frac{q}{m}\left(E_p+\frac{v_p}{c}\times B_p\right)
\end{split}
\end{align}

Где поля, действующие на частицу вычисляются усреднением по объему

\begin{equation}
	E_p = \frac{1}{V_c}\int d^3 x S(x-x_p)E(x)
\end{equation}

Временная эволюция электрического и магнитного полей вычисляется с помощью численного решения уравнений Максвела, с учетом тока, создаваемого частицами. Сложность заключается в том, что если вычислять ток, усредняя скорость движения частиц в данной ячейке, не будет выполняться уравнение непрерывности для плотности заряда. В результате потребуется решение уравнения Пуассона для коррекции электрического поля. Чтобы избежать этой проблемы, в коде SMILEI используется метод декомпозиции тока на составляющие, связанные с движением вдоль каждой из осей, предложенный Езиркеповым \cite{Esirkepov}, позволяющий автоматически сохранять уравнение непрерывности.

Еще одна существенна проблема Particle-in-Cell кодов заключается в том, что численное дисперсионное соотношение для свободных электромагнитных волн выглядит (в простой конечно-разностной схеме, предложенной Yee \cite{Yee}) выглядит так:
 \begin{equation} \label{dispersion}
 	\begin{split}
 	\left(\frac{1}{c\Delta t}sin\left(\frac{\omega \Delta t}{2}\right)\right)^2 = \left(\frac{1}{\Delta x}sin\left(\frac{k_x \Delta x}{2}\right)\right)^2 +\\ \left(\frac{1}{\Delta y}sin\left(\frac{k_y \Delta y}{2}\right)\right)^2 + \left(\frac{1}{\Delta z}sin\left(\frac{k_z \Delta z}{2}\right)\right)^2
 	\end{split}
 \end{equation}

в отличие от физического соотношения $\omega=\sqrt{k_x^2 + k_y^2 + k_z^2}$, и переходит в него только при бесконечно малых шагах по времени и координате. Это приводит к тому, что высокочастотные гармоники обладают фазовой скоростью меньшей чем скорость света. При наличии релятивистских частиц, обладающих большими скоростями, такие гармоники будут усиливаться за счет численной черенковской неустойчивости \cite{Godfrey1974}, приводя к нефизичному нагреву плазмы и нарушению закона сохранения энергии.

Существует множество способов борьбы с численной черенковской неустойчивостью. Самый естесственный - изменение процедуры численного вычисления производных полей в уравнениях Максвелла, чтобы получить дисперсионное соотношение, более близкое к физическому. Примеры таких схем можно найти в работе Гринвуда \cite{Greenwood2004}. Недостатком этого метода является повышащаяся сложность вычислений, возникающие проблемы с сохранением заряда и граничными условиями. Другим направлением подавления черенковской неустойчивости является использование различных фильтров. Во-первых можно использовать простейшее сглаживание по соседним ячейкам, что будет приводить к уменьшению высокочастотных гармоник. Во-вторых - применить преобразование Фурье для полноценного устранения высокочастотных гармоник. Но быстрое дискретное преобразование Фурье затруднительно использовать при распределенных многопроцессорных вычислениях. В-третьих используются фильтры, связанные с шагом по времени - фильтр Годфри \cite{Godfrey1980}, модифицирующий уравнение для вычисления электрического поля:

\begin{equation}
	\frac{E^{n+1}-E^n}{c \Delta t} = \nabla \times \left(\alpha_1 B^{n+3/2}+\alpha_2 B^{n+1/2} + \alpha_3 B^{n-1/2}\right)
\end{equation}

где $\alpha_1, \alpha_2, \alpha_3$ - постоянные коэффициенты, в сумме дающие единицу. Фильтр Фридмана \cite{Friedman1990} модифицирует уравнение для вычисления магнитного поля:

\begin{equation}
	\begin{split}
	\frac{B^{n+1/2}-B^{n-1/2}}{c \Delta t} = -\nabla \times \left(\left(1+\frac{\theta}{2}\right)E^n-\theta\left(1-\frac{\theta}{2}\right)E^{n-1}+\right.\\ \left.\frac{1}{2}\left(1-\theta\right)^2 \theta\left(E^{n-2} + \theta E^{n-3}\right)\right)
	\end{split}
\end{equation}

где $\theta$ - параметр от 0 до 1. Фильтр Годфри использует магнитное поле в следующий момент времени, что приводит к необходимости итеративного решения неявного уравнения, поэтому фильтр Фридмана является предпочтительным.

%\begin{equation}
%	J(x) = \sum_s q_s \int d^3 p v(p) F_s(x,p) = \sum_s %\sum_p q_s v(p) \frac{\omega_p}{V_c} \int d^3 x S(x-x_p)
%\end{equation}

\FloatBarrier
\section{Выводы}
 

Получены следующие результаты:
\begin{enumerate}
\item 
\item 
\end{enumerate}


\clearpage