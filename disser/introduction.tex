\chapter*{Введение}					
\addcontentsline{toc}{chapter}{Введение}	%добавляем в оглавление

\section*{Быстрые голубые оптические транзиенты}
\addcontentsline{toc}{section}{Космические гамма-всплески}	%добавляем в оглавление



\section*{Актуальность темы диссертации}
\addcontentsline{toc}{section}{Актуальность темы диссертации}	%добавляем в оглавление


\section*{Цели работы}
\addcontentsline{toc}{section}{Цели работы}	%добавляем в оглавление

Для достижения поставленной цели решаются следующие задачи:
\begin{enumerate}
\item 
\end{enumerate}

\section*{Научная новизна}
\addcontentsline{toc}{section}{Научная новизна}	%добавляем в оглавление

Следующие основные результаты получены впервые:
\begin{enumerate}
\item 
\end{enumerate}

\section*{Научная и практическая значимость}
\addcontentsline{toc}{section}{Научная и практическая значимость}	%добавляем в оглавление

\begin{enumerate}
\item 
\end{enumerate}

\section*{Основные положения, выносимые на защиту}
\addcontentsline{toc}{section}{Основные положения, выносимые на защиту}

\begin{enumerate}
\item 
\end{enumerate}

\section*{Апробация работы и публикации}
Результаты, вошедшие в диссертацию, получены в период с 2014 по 2022~годы и 
опубликованы в 4~статьях в реферируемых журналах и в тезисах 5 конференций. 

Статьи в рецензируемых изданиях:
\begin{enumerate}
\item %V.\,D.\,Pal'shin, K.\,Hurley, D.\,S.\,Svinkin et al. Interplanetary Network Localizations of
%Konus Short Gamma-Ray Bursts // Astrophys.~J.~Suppl. 2013. Vol.~207. id~38;


\end{enumerate}

Результаты докладывались на всероссийских и международных конференциях: 
\begin{enumerate}
\item 
\end{enumerate}
и на семинарах сектора теоретической астрофизики ФТИ~им.~А.~Ф.~Иоффе и ГАИШ МГУ.

\section*{Личный вклад}
\addcontentsline{toc}{section}{Личный вклад}


\section*{Структура диссертации}
\addcontentsline{toc}{section}{Структура диссертации}
Диссертация состоит из введения, 5 глав, заключения и библиографии.
Общий объем диссертации 155 страниц, включая 33 рисунка, 13 таблиц. 
Библиография включает 206 наименований на 18 страницах.

Во \textbf{введении} приведено 
Также сформулированы основные результаты и положения, выносимые на защиту, и приведен
список работ, в которых опубликованы основные результаты диссертации.

\textbf{Глава~\ref{FBOT}} посвящена

  
\enlargethispage{4\baselineskip}
\textbf{Заключение} содержит краткий обзор полученных в диссертации результатов.

\clearpage